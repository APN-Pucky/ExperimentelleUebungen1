% Autor: Leonhard Segger, Alexander Neuwirth
% Datum: 2017-10-30
\documentclass[
	% Papierformat
	a4paper,
	% Schriftgröße (beliebige Größen mit „fontsize=Xpt“)
	12pt,
	% Schreibt die Papiergröße korrekt ins Ausgabedokument
	pagesize,
	% Sprache für z.B. Babel
	ngerman
]{scrartcl}

% Achtung: Die Reihenfolge der Pakete kann (leider) wichtig sein!
% Insbesondere sollten (so wie hier) babel, fontenc und inputenc (in dieser
% Reihenfolge) als Erstes und hyperref und cleveref (Reihenfolge auch hier
% beachten) als Letztes geladen werden!

% Silbentrennung etc.; Sprache wird durch Option bei \documentclass festgelegt
\usepackage{babel}
% Verwendung der Zeichentabelle T1 (Sonderzeichen etc.)
\usepackage[T1]{fontenc}
% Legt die Zeichenkodierung der Eingabedatei fest, z.B. UTF-8
\usepackage[utf8]{inputenc}
% Schriftart
\usepackage{lmodern}
% Zusätzliche Sonderzeichen
\usepackage{textcomp}

% Mathepaket (intlimits: Grenzen über/unter Integralzeichen)
\usepackage[intlimits]{amsmath}
% Ermöglicht die Nutzung von \SI{Zahl}{Einheit} u.a.
\usepackage{siunitx}
% Zum flexiblen Einbinden von Grafiken (\includegraphics)
\usepackage{graphicx}
% Abbildungen im Fließtext
\usepackage{wrapfig}
% Abbildungen nebeneinander (subfigure, subtable)
\usepackage{subcaption}
% Funktionen für Anführungszeichen
\usepackage{csquotes}
% Zitieren, Bibliographie
\usepackage{biblatex}


% Zur Darstellung von Webadressen
\usepackage{url}
%chemische Formeln
\usepackage[version=4]{mhchem}
% siunitx: Deutsche Ausgabe, Messfehler getrennt mit ± ausgeben
\usepackage{floatrow}
\floatsetup[table]{capposition=top}
% Verlinkt Textstellen im PDF-Dokument
\usepackage[unicode]{hyperref}
% "Schlaue" Referenzen (nach hyperref laden!)
\usepackage{cleveref}
\sisetup{
	locale=DE,
	separate-uncertainty
}
%\bibliography{6Mi_M3_29-11-2017_References}

\begin{document}
	
	\begin{titlepage}
		\centering
		{\scshape\LARGE Versuchsbericht zu \par}
		\vspace{1cm}
		{\scshape\huge M5 - Jo-Jo und Kreisel\par}
		\vspace{2.5cm}
		{\LARGE Gruppe 6Mi \par}
		\vspace{0.5cm}
		
		{\large Alexander Neuwirth (E-Mail: a\_neuw01@wwu.de) \par}
		{\large Leonhard Segger (E-Mail: l\_segg03@uni-muenster.de) \par}
		\vfill
		
		durchgeführt am 13.12.2017\par
		betreut von\par
		{\large Kristina Mühlenstrodt} %TODO Anpassen
		
		\vfill
		
		{\large \today\par}
	\end{titlepage}
	\tableofcontents
	\newpage

	%TODO mehr TODO in Default	

	\section{Kurzfassung}
	%TODO Hypothese	
	%TODO Ergenisse
	%TODO Was wurde gemacht
	
	\section{Methoden}
	%TODO Bilder von der Website klauen
	
	\section{Ergebnisse und Diskussion}
	%TODO Datenanalyse -> Überschrift?
	%TODO Unsicherheiten
	

	\subsection{Beobachtung}
	\subsubsection{Fallrad}
	
	\subsubsection*{Berechnung des Trägheitsmoments}
	Das Trägheitsmoments des Fallrads setzt sich aus den Trägheitsmomenten der einzelnen Komponenten zusammen.  %TODO Fallrads oder rades,... Duden sagt beides
	In \cref{Tabelle_Traegheitsmomente_Zylinder} sind Volumen und Trägheitsmoment von Zylindern aufgeführt. 
	Die Masse $m$ ergibt sich jeweils aus 
	\begin{equation}
		m = M \frac{V}{V_\text{ges}} 
	\end{equation}
	wobei $M$ die Masse des gesamten Fallrads und $V_\text{ges}$ entsprechend das gesamte Volumen ist. Es wird davon ausgegangen, dass der Stoff homogen ist. % => Dichte konstant + gleiche Zusammensetzung

	\begin{table}[tb]
		\centering
		\begin{tabular}{ r |  c | c | c}
			Zylinder& Volumen V &Rotationsachse & Trägheitsmoment $J$\\ \hline
			Vollzylinder& $\pi lr^2$ &Symmetrieachse & $\frac{1}{2} m r^2$ \\
			Vollzylinder& $\pi lr^2$&Querachse & $\frac{1}{4} m r^2 + \frac{1}{12} m l^2$ \\
			Hohlzylinder& $\pi l(r_2^2 - r_1^2)$&Symmetrieachse & $\frac{1}{2} m (r_1^2 + r_2^2)$ \\
		\end{tabular}
		\caption{Trägheitsmomente von (Hohl-)Zylindern zu verschiedenen Achsen.}
		\label{Tabelle_Traegheitsmomente_Zylinder} 
	\end{table}

	\begin{table}[tb]
		\centering
		\begin{tabular}{ r |  c | c | c | c }
			& Längeo $L$ & Unsicherheit  & Radius $R$& Unsicherheit\\ \hline
			Speiche &\SI{15,576}{cm} & \SI{0,023}{cm} & \SI{0,407}{cm} & \SI{0,006}{cm}\\
			Achse & \SI{20,21}{cm} & \SI{0,012}{cm} & \SI{0,405}{cm} & \SI{0,006}{cm}\\
			Rad außen&- & -& \SI{9,007}{cm} &  \SI{0,001}{cm} \\
			Rad innen&- & -& \SI{7,788}{cm} &  \SI{0,012}{cm} \\
			Dicke Rad & \SI{1,15}{cm} & \SI{0,004}{cm} &- & -
		\end{tabular}
		\caption{ Gemessene Längen.}
		\label{Tabelle_Laenge} 
	\end{table}
	Es folgt das Trägheitsmoment mit einer Gesamtmasse $M$ von \SI{0,76807 \pm 0,000028}{kg}:
	\begin{equation}
		J = \frac{1}{2}\frac{M}{2V_S+V_A+V_R} ( V_A R_A^2 + V_S (R_S^2 + \frac{1}{3} L_S^2) + V_R (R_\text{Rad,Außen}^2 + R_\text{Rad,Innen}^2)
	\end{equation}

	\begin{equation}
		u(y) = \sqrt{  \sum_{i=0}^{N} \left( \frac{\partial f}{\partial x_i}u(x_i)\right)^2  }
		\label{Partielle_Unsicherheiten}
	\end{equation}
	Beim Einsetzten aller Größen ergibt sich ein Trägheitsmoment von J = \SI{42.550 \pm 0,1875}{kg/cm^2} mit einer relativen Abweichung von 0,441\%.


	\subsection{Diskussion}
	%TODO Bezug/Nutzten oder sonst was
	
	\section{Schlussfolgerung}
	%TODO Rückgriff auf Hypothese
	
	%TODO Quellen zitieren, Websiten mit Zugriffsdatum
	%TODO Verweise auf das Laborbuch (sind erlaubt)
	%TODO Tabelle + Bilder mit Beschriftung
	\section{Beantwortung der Aufgaben zur Vorbereitung}
	\begin{enumerate}
		\item 
			\begin{align}
				0 = \frac{dE}{dt} &= \frac{d}{dt}(\frac{1}{2} m v^2 + \frac{1}{2} J_S \omega^2 - mgh) \\
				&= mva + \frac{J_S}{R^2}va - mgv\\
				\frac{mg}{a} &= m + \frac{J_S}{R^2} \\
				\Rightarrow a(t) &= g \frac{mR^2}{mR^2+J_S} \\
				\Rightarrow h(t) &= \frac{1}{2} g\frac{mR^2}{mR^2+J_S} t^2 + v_0t + h_0
			\end{align}
		\item
			Die Kraft mit der das abrollende Rad an der Aufhängevorichtung zieht ergibt sich aus 
			\begin{equation}
				F = ma
			\end{equation}
			und beträgt folglich $mg\frac{mR^2}{mR^2+J_S}$. 
			Dass die Kraft, bzw. Beschleunigung, konstant ist, ist auch in Abbildung 2 der Einführung zum Versuch dargestellt. 
			Der Unterschied zur Gewichtskraft des Rades besteht in dem Faktor $\frac{mR^2}{mR^2+J_S}$, welcher stets kleiner als 1 ist, somit fällt das Rad langsamer als im freien Fall.
		\item
			Die Kraft wirkt nach wie vor in die gleiche Richtung mit gleichem Betrag.
	\end{enumerate}
	%\printbibliography
\end{document}
