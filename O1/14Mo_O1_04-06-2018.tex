% Autor: Leonhard Segger, Alexander Neuwirth
% Datum: 2017-10-30
\documentclass[
	% Papierformat
	a4paper,
	% Schriftgröße (beliebige Größen mit „fontsize=Xpt“)
	12pt,
	% Schreibt die Papiergröße korrekt ins Ausgabedokument
	pagesize,
	% Sprache für z.B. Babel
	ngerman
]{scrartcl}

% Achtung: Die Reihenfolge der Pakete kann (leider) wichtig sein!
% Insbesondere sollten (so wie hier) babel, fontenc und inputenc (in dieser
% Reihenfolge) als Erstes und hyperref und cleveref (Reihenfolge auch hier
% beachten) als Letztes geladen werden!

% Silbentrennung etc.; Sprache wird durch Option bei \documentclass festgelegt
\usepackage{babel}
% Verwendung der Zeichentabelle T1 (Sonderzeichen etc.)
\usepackage[T1]{fontenc}
% Legt die Zeichenkodierung der Eingabedatei fest, z.B. UTF-8
\usepackage[utf8]{inputenc}
% Schriftart
\usepackage{lmodern}
% Zusätzliche Sonderzeichen
\usepackage{textcomp}

% Mathepaket (intlimits: Grenzen über/unter Integralzeichen)
\usepackage[intlimits]{amsmath}
% Ermöglicht die Nutzung von \SI{Zahl}{Einheit} u.a.
\usepackage{siunitx}
% Zum flexiblen Einbinden von Grafiken (\includegraphics)
\usepackage{graphicx}
% Abbildungen im Fließtext
\usepackage{wrapfig}
% Abbildungen nebeneinander (subfigure, subtable)
\usepackage{subcaption}
% Funktionen für Anführungszeichen
\usepackage{csquotes}
\MakeOuterQuote{"}
% Zitieren, Bibliographie
\usepackage{biblatex}


% Zur Darstellung von Webadressen
\usepackage{url}
%chemische Formeln
\usepackage[version=4]{mhchem}
% siunitx: Deutsche Ausgabe, Messfehler getrennt mit ± ausgeben
\usepackage{floatrow}
\floatsetup[table]{capposition=top}
\usepackage{float}
% Verlinkt Textstellen im PDF-Dokument
\usepackage[unicode]{hyperref}
% "Schlaue" Referenzen (nach hyperref laden!)
\usepackage{cleveref}
\sisetup{
	locale=DE,
	separate-uncertainty
}
%\bibliography{14Mo_O1_04-06-2018_References}
%TODO anpassen

\begin{document}
	
	\begin{titlepage}
		\centering
		{\scshape\LARGE Versuchsbericht zu \par}
		\vspace{1cm}
		{\scshape\huge O1 - Geometrische Optik \par}
		\vspace{2.5cm}
		{\LARGE Gruppe 14Mo \par}
		\vspace{0.5cm}
		
		{\large Alexander Neuwirth (E-Mail: a\_neuw01@wwu.de) \par}
		{\large Leonhard Segger (E-Mail: l\_segg03@uni-muenster.de) \par}
		\vfill
		
		durchgeführt am 04.04.2018\par
		betreut von\par
		{\large Helge Gehring} 
		
		\vfill
		
		{\large \today\par}
	\end{titlepage}
	\tableofcontents
	\newpage

	%TODO mehr TODO in Default	

	\section{Kurzfassung}
	%TODO Hypothese	und deren Ergebnis, wenn Hypothese ist, dass nur Theorie erfüllt, sagen: Erwartung: Theorie aus einführung (mit reflink) erfüllt
	%TODO Ergebnisse, auch Zahlen, mindestens wenn's halbwegs Sinn ergibt
	%TODO Was wurde gemacht
	%TODO manche leute wollen Passiv oder "man", manche nicht
	
	\section{Methoden}
	%TODO Bilder von der Website klauen
	
	\section{Ergebnisse und Diskussion}
	%TODO Unsicherheiten
	

	\subsection{Beobachtung}
	%TODO Diesmal würd ich die Unsicherheiten on the run machen
	\subsubsection{Demonstrationsversuch}
	\subsubsection{Prisma} %TODO Unsicherhiet ungenau weil Punkt, nur circa minimal, ungenau anfang enede
	\subsubsection{Brechungsindex von Wasser}
	\subsubsection{Brennweite der Sammellinse}
	\subsubsection{Brennweite der Streulinse}
	\subsubsection{Strahlaufweitung und Sammellinse}
	%TODO Einflüsse von veränderten Parametern auf Messung
	\subsection{Datenanalyse}
	%TODO Berechung nach Aufgabenstellung
	\subsubsection{Prisma}
	In der Einleitung wurde \cref{eq_prisma} zur Bestimmung des Brechungsindex des Prismamaterials, bei einer minimalen Ablenkung $\delta_m$, aufgeührt.
	\begin{equation}
		n = \frac{\sin\left[(\delta_m+\alpha)/2\right]}{\sin\left(\alpha/2\right)}
		\label{eq_prisma}
	\end{equation}
	\begin{equation}
		u(n) = u(\delta_m) \cdot \left|\frac{\sin(a/2)\cos[(a+\delta_m)/2]}{\cos(\alpha)-1}\right|
	\end{equation}

	Dabei wurde in einem Abstand $d$ eine orthogonale Auslenkung $a$ gemessen.
	Der Apexwinkel $\alpha$ beträgt $\SI{60}{\degree}$.
	Es folgt eine minimale Auslenkung $\delta_m = \arctan (a/d)$.
	Die aus den Messungen folgenden Werte sind in \cref{tab_prisma} aufgelistet.
	%TODO Laser Wellenlänge dazu?
	\begin{table}[H]
		\centering
		\begin{tabular}{ c | c | c | c | c}
			Laser & Auslenkung $a$  & Abstand $d$ & $\delta_m$ & $n$ \\ \hline
			rot & \SI{13,23 +- 0,14}{cm}&\SI{12+-0,2}{cm} & \SI{0,8341 +- 0,0098}{rad}&\SI{1,616 +- 0,006}{}\\
			blau & \SI{14,82 +- 0,14}{cm}&\SI{12+-0,2}{cm}& \SI{0,8901 +- 0,0094}{rad}&\SI{1,648 +- 0,005}{} \\
		\end{tabular}
		\caption{Aus gemessener Auslenkung des Lichtstrahls und Abstand des Lineals lässt sich der Ablenkungswinkel $\delta$ bestimmen. Der Brechungsindex des Prismas folgt widerum aus dem minimalen Ablenkungswinkel $\delta_m$.}
		\label{tab_prisma}
	\end{table}

	\subsubsection{Brechungsindex von Wasser}
	Das Snelliussche Brechungsgesetz lautet:
	\begin{equation}
		n_i \cdot \sin(\vartheta_i) = n_t \cdot \sin(\vartheta_t)
		\label{eq_snellius}
	\end{equation}
	Somit kann der Brechungsindex $n_\text{Wasser}$  mit 
	\begin{equation}
		n_\text{Wasser} = n_\text{Luft}\frac{\sin(\vartheta_\text{Luft})}{\sin(\vartheta_\text{Wasser})}
	\end{equation}
	\begin{equation}
		u(n_\text{Wasser}) = n_\text{Luft} \sqrt{\left(\frac{\cos(\vartheta_\text{Luft})}{\sin(\vartheta_\text{Wasser})} u(\vartheta_\text{Luft})\right)^2 + \left(\frac{\sin(\vartheta_\text{Luft})\cos( \vartheta_\text{Wasser})}{\sin(\vartheta_\text{Wasser})^2}u(\vartheta_\text{Wasser})\right)^2}
	\end{equation}

	gemessen werden.
	Die Messwerte sowie resultierende Brechungsindizes sind in \cref{tab_wasser} aufgeführt.
	Gemittelt ergibt sich ein Brechungsindex für Wasser bei rotem Licht von \SI{1,309 +- 0,010}{} und bei blauem \SI{1,318 +- 0,012}{}.
	%TODO Laser Wellenlänge dazu?
	\begin{table}[H]
		\centering
		\begin{tabular}{ c | c | c | c | c}
			Laser & Ordnung & $\vartheta_\text{Luft}$ & $\vartheta_\text{Wasser}$ & $n_\text{Wasser}$ \\ \hline
			rot & -2 &\SI{52,8 +- 0,3}{\degree} & \SI{37,5 +-0,3}{\degree} & \SI{1,308 +- 0,010}{}\\
			& -1 &\SI{23,8 +- 0,3}{\degree} & \SI{18+-0,3}{\degree} & \SI{1,306 +- 0,026}{}\\
			& 1 &\SI{24 +- 0,3}{\degree} & \SI{18+-0,3}{\degree} & \SI{1,316 +- 0,026}{}\\
			& 2 &\SI{53,5 +- 0,3}{\degree} & \SI{38+-0,3}{\degree} & \SI{1,306 +- 0,010}{}\\ \hline
			blau & -3 & \SI{48,0 +- 0,3}{\degree}&\SI{34,5+-0,3}{\degree}& \SI{1,312 +- 0,012}{} \\
			& -2 & \SI{30,0 +- 0,3}{\degree}&\SI{22,2+-0,3}{\degree}& \SI{1,323 +- 0,021}{} \\
			& -1 & \SI{14,5 +- 0,3}{\degree}&\SI{10,9+-0,3}{\degree}& \SI{1,324 +- 0,045}{} \\
			& 1 & \SI{14,5 +- 0,3}{\degree}&\SI{11+-0,3}{\degree}& \SI{1,312 +- 0,044}{} \\
			& 2 & \SI{30,0 +- 0,3}{\degree}&\SI{22,2+-0,3}{\degree}& \SI{1,323 +- 0,021}{} \\
			& 3 & \SI{48,5 +- 0,3}{\degree}&\SI{34,8+-0,3}{\degree}& \SI{1,312 +- 0,012}{} \\

		\end{tabular}
		\caption{}
		\label{tab_wasser}
	\end{table}
	
	\subsection{Diskussion}
	%TODO Bezug/Nutzten oder sonst was
	%TODO auch hier die Hypothese wiederholen
	%TODO keine Messwerte hier, nach manchen Menschen, zumindest "direkt" erstellte Diagramme net hier, auch wenn Lesbarkeit-bla
	
	\section{Schlussfolgerung}
	%TODO Rückgriff auf Hypothese und drittes Nennen dieser
	
	%TODO Quellen zitieren, Websiten mit Zugriffsdatum
	%TODO Verweise auf das Laborbuch (sind erlaubt)
	%TODO Tabelle + Bilder mit Beschriftung
	%\printbibliography
\end{document}
