% Autor: Leonhard Segger, Alexander Neuwirth
% Datum: 2017-10-30
\documentclass[
	% Papierformat
	a4paper,
	% Schriftgröße (beliebige Größen mit „fontsize=Xpt“)
	12pt,
	% Schreibt die Papiergröße korrekt ins Ausgabedokument
	pagesize,
	% Sprache für z.B. Babel
	ngerman
]{scrartcl}

% Achtung: Die Reihenfolge der Pakete kann (leider) wichtig sein!
% Insbesondere sollten (so wie hier) babel, fontenc und inputenc (in dieser
% Reihenfolge) als Erstes und hyperref und cleveref (Reihenfolge auch hier
% beachten) als Letztes geladen werden!

% Silbentrennung etc.; Sprache wird durch Option bei \documentclass festgelegt
\usepackage{babel}
% Verwendung der Zeichentabelle T1 (Sonderzeichen etc.)
\usepackage[T1]{fontenc}
% Legt die Zeichenkodierung der Eingabedatei fest, z.B. UTF-8
\usepackage[utf8]{inputenc}
% Schriftart
\usepackage{lmodern}
% Zusätzliche Sonderzeichen
\usepackage{textcomp}

% Mathepaket (intlimits: Grenzen über/unter Integralzeichen)
\usepackage[intlimits]{amsmath}
% Ermöglicht die Nutzung von \SI{Zahl}{Einheit} u.a.
\usepackage{siunitx}
% Zum flexiblen Einbinden von Grafiken (\includegraphics)
\usepackage{graphicx}
% Abbildungen im Fließtext
\usepackage{wrapfig}
% Abbildungen nebeneinander (subfigure, subtable)
\usepackage{subcaption}
% Funktionen für Anführungszeichen
\usepackage{csquotes}
% Zitieren, Bibliographie
\usepackage{biblatex}


% Zur Darstellung von Webadressen
\usepackage{url}
%chemische Formeln
\usepackage[version=4]{mhchem}
% siunitx: Deutsche Ausgabe, Messfehler getrennt mit ± ausgeben
\usepackage{floatrow}
\floatsetup[table]{capposition=top}
% Verlinkt Textstellen im PDF-Dokument
\usepackage[unicode]{hyperref}
% "Schlaue" Referenzen (nach hyperref laden!)
\usepackage{cleveref}
\sisetup{
	locale=DE,
	separate-uncertainty
}
%\bibliography{6Mi_M3_29-11-2017_References}

\begin{document}
	
	\begin{titlepage}
		\centering
		{\scshape\LARGE Versuchsbericht zu \par}
		\vspace{1cm}
		{\scshape\huge E1 - Gleich- und Wechselstrom\par}
		\vspace{2.5cm}
		{\LARGE Gruppe 6Mi \par}
		\vspace{0.5cm}
		
		{\large Alexander Neuwirth (E-Mail: a\_neuw01@wwu.de) \par}
		{\large Leonhard Segger (E-Mail: l\_segg03@uni-muenster.de) \par}
		\vfill
		
		durchgeführt am 20.12.2017\par 
		betreut von\par
		{\large Philipp Eickholt} %Anpassen war das Fabian Schöttke? Nein    im Laborbuch steht nichts                         
		
		\vfill
		
		{\large \today\par}
	\end{titlepage}
	\tableofcontents
	\newpage


	\section{Kurzfassung}
	%TODO Hypothese	und deren Ergebnis
	%TODO Ergebnisse, auch Zahlen, mindestens wenn's halbwegs Sinn ergibt
	%TODO Was wurde gemacht
	Es wurden zwei Experimente zu Gleich- und Wechselstrom bzw. dem Innenwiderstand von Stromquellen durchgeführt.
	Im ersten Experiment wurde durch Messung der Klemmspannung eines Akkumulators in Abhängigkeit vom Außenwiderstand in einem einfachen Stromkreis untersucht.
	Da hier der Innenwiderstand durch einen angelöteten Widerstand künstlich erhöht war, war zu erwarten, dass der berechnete Gesamtinnenwiderstand sich nur geringfügig (nämlich um den tatsächlichen Innenwiderstand der Zelle) von dem auf dem Widerstand angegebenen Widerstand unterscheidet. %TODO Ergebnis
	\par 
	Im zweiten Experiment wurde die Leistungsaufnahme verschiedener Verbraucher in Abhängigkeit von der Größe und Form (Gleich- oder Wechselstrom) der angelegten Spannung untersucht.
	Hieraus wurde auch Wirkwiderstand, Innenwiderstand und Phasenwinkel einer Spule sowie Betrag und Phase des Wechselstromwiderstandes der Kombination von Spule und Kondensator bestimmt. %TODO ist das true?
	Hierbei war zu erwarten, dass die Messungen die theoretisch erwarteten Werte für Widerstände, Spulen und Kondensatoren widerspiegeln. %TODO Ergebnis
	
	
	\section{Methoden}
	%TODO Bilder von der Website klauen
	Zunächst wurde ein einfacher Stromkreis aufgebaut, mit dem der Innenwiderstand von Akkumulatorzellen gemessen werden konnte.
	Dieser bestand aus einem veränderlichen Lastwiderstand, der Stromquelle und einem Spannungsmessgerät, mit dem die Klemmspannung der Stromquelle erfasst wurde.
	Dann wurde der Lastwiderstand verändert und die Klemmspannung über der Stromquelle gemessen.
	Dies wurde drei mal mit verschiedenen Stromquellen durchgeführt.
	Hier wurde zunächst eine einzelne Akkumulatorzelle, dann drei Zellen parallel und zuletzt drei Zellen in Reihenschaltung verwendet.
	Dabei hatten die Zellen einen künstlich (durch einen eingebauten Widerstand) erhöhten Innenwiderstand.
	Dieser Widerstand wurde abgelesen.
	\par
	
	\begin{figure}[tb]
		\includegraphics[width=0.8\textwidth]{Schaltkreis2}
		\centering
		\caption{Der Schaltkreis, der zur Messung der Leistungsaufnahme verschiedener Verbraucher benutzt wurde.}
		\label{SK2}
		\centering
	\end{figure}
	
	Dann wurde die in \cref{SK2} dargestellte Schaltung zur Messung der Leistungsaufnahme verschiedener Verbraucher bei Gleich- und Wechselstrom aufgebaut.
	Hierfür wurde zunächst für zwei verschiedene Voltmeter deren Verlustleistung in Abhängigkeit von Gleich- bzw. Wechselspannung beobachtet, indem die Schaltung in \cref{SK2} ohne Verbraucher verwendet wurde, um einer Entscheidung treffen zu können, welches Messgerät im Folgenden verwendet werden sollte und inwiefern dessen Verlustleistung berücksichtigt werden muss.
	Dann wurde für ein festes $ R_2 $ bei Gleich- und Wechselstrom Spannung, Stromstärke und Leistung für fünf verschiedene $ R_1 $ gemessen.
	Hierbei wurde $ R_2 $ so gewählt, das ein möglichst großer Messbereich der Messgeräte genutzt werden konnte.
	Dann wurde diese Messung für eine Spule sowie eine Spule mit Kondensator wiederholt. 
	
	\section{Ergebnisse und Diskussion}
	%TODO Datenanalyse -> Überschrift?
	%TODO Unsicherheiten
	%TODO Spannungs messung ohne Strom fluss wurde Widerstan auf 2000 Ohm geschätz -> Satz
	%TODO Unsicherheiten auflisten woher warum....
	\subsection{Beobachtung}
	\subsubsection{Innenwiderstand}
	In \cref{Spannung1} ist die Klemmspannung gegen den Strom, der sich aus $I = U/R$ ergeben hat, aufgetragen. 
	Es wurde ein linearer Fit durchgeführt, da nach der Theorie ein linearer Zusammenhang besteht. 
	Die Steigung der Geraden ist der (negative) Innenwiderstand $R_\text{i} = \SI{27,19 \pm 0,47}{\Omega}$.

	Trägt man die Leistung gegen den Außenwiderstand, ist zuerwarten, dass (genau) ein Maximum bei $R_\text{i}  R_\text{a}$ liegt.
	\cref{Leistung1} stellt dies und einen Fit mit dem \enquote{Scaled Levenberg-Marquardt}-Algorithmus, welcher die Methode der kleinsten Quadrate verwendet, dar. 
	Die Funktion des Fits ist:
	\begin{equation}
		f(x)=a\frac{x}{(x+b)^2}
	\end{equation}
	Es ergibt sich ein Parameter $b = \SI{29,51}{}$ ohne Unsicherheit, desshalb haben wir diese als relative Unsicherheit mit 2\% abgeschätzt. 
	Folglich ist $R_\text{i} = \SI{29,51 \pm 0,59}{\Omega}$.

	Analog kann man aus \crefrange{Spannung3Reihe}{Leistung3Parallel} die Innenwiderstände für drei parallel, bzw. in Reihe, geschaltete Akkus erhalten.
	In \cref{Tabelle_Innenwiderstaende} sind die ermittelten Innenwiderstände aufgelistet. 
	Aus diesen Widerständen lässt der Innenwiderstand eines einzelnen Akkus bestimmen.
	\cref{Tabelle_Innenwiderstaende2} zeigt diese. %TODO ihhhhh zeigt


	\subsubsection*{Zusatzfrage}
	Eine Stromquelle soll einen möglichst konstanten Strom liefern. Das wird erreicht durch einen möglichst hohen Innenwiderstand, also eine Reihenschaltung der Spannungsquellen. Für eine Spannungsquelle bietet sich eine Parallelschaltung an, da der Innenwiderstand gering und somit die Spannung konstant gehalten werden kann.
	\begin{table}[tb]
		\centering
		\begin{tabular}{ r | c | c | c}
			Innenwiderstand& Ein Akku & 3 Akkus Reihe & 3 Akkus Parallel \\ \hline
			aus Klemmspannung& \SI{27,19 \pm 0,47 }{\Omega}& \SI{81,24 \pm 1,06 }{\Omega}&  \SI{9,73 \pm 0,20 }{\Omega} \\
			aus Leistung & \SI{29,51 \pm 0,59 }{\Omega}&  \SI{77,53 \pm 1,55 }{\Omega}&  \SI{9,79 \pm 0,19 }{\Omega}\\

		\end{tabular}
		\caption{Gemessener Innenwiderstand.} %TODO mehr hier
		\label{Tabelle_Innenwiderstaende} 
	\end{table}
	\begin{table}[tb]
		\centering
		\begin{tabular}{ r | c | c | c}
			Innenwiderstand& Ein Akku & Akku Reihe & Akku Parallel \\ \hline
			aus Klemmspannung& \SI{27,19 \pm 0,47 }{\Omega}& \SI{27,08 \pm 0,35 }{\Omega}&  \SI{29,19 \pm 0,60 }{\Omega} \\
			aus Leistung & \SI{29,51 \pm 0,59 }{\Omega}&  \SI{25,84 \pm 0,52 }{\Omega}&  \SI{29,37 \pm 0,57 }{\Omega}\\

		\end{tabular}
		\caption{Gemessener Innenwiderstand.} %TODO besser Beschreibung
		\label{Tabelle_Innenwiderstaende2} 
	\end{table}

	\begin{figure}[tb]
		\includegraphics[width=1\textwidth]{Spannung1}
		\centering
		\caption{Die gemessene Klemmspannung bei einem Akku ist gegen den Strom aufgetragen.}
		\label{Spannung1}
		\centering
	\end{figure} 
	\begin{figure}[tb]
		\includegraphics[width=1\textwidth]{Leistung1}
		\centering
		\caption{Die gemessene Leistung bei einem Akku ist gegen den Außenwiderstand aufgetragen.}
		\label{Leistung1}
		\centering
	\end{figure}
	\begin{figure}[tb]
		\includegraphics[width=1\textwidth]{Spannung3Reihe}
		\centering
		\caption{Die gemessene Klemmspannung bei drei in Reihe geschateten Akkus ist gegen den Strom aufgetragen.}
		\label{Spannung3Reihe}
		\centering
	\end{figure} 
	\begin{figure}[tb]
		\includegraphics[width=1\textwidth]{Leistung3Reihe}
		\centering
		\caption{Die gemessene Leistung bei drei in Reie geschalteten Akkus ist gegen den Außenwiderstand aufgetragen.}
		\label{Leistung3Reihe}
		\centering
	\end{figure}
	
	\begin{figure}[tb]
		\includegraphics[width=1\textwidth]{Spannung3Parallel}
		\centering
		\caption{Die gemessene Klemmspannung bei 3 parallelen Akkus ist gegen den Strom aufgetragen.}
		\label{Spannung3Parallel}
		\centering
	\end{figure}

	\begin{figure}[tb]
		\includegraphics[width=1\textwidth]{Leistung3Parallel}
		\centering
		\caption{Die gemessene Leistung bei drei parallelen Akkus ist gegen den Außenwiderstand aufgetragen.}
		\label{Leistung3Parallel}
		\centering
	\end{figure}
	%TODO differenzieren zwischen wechsel und gleich in den Graph Beschreibungen

	\subsubsection{Gleich- und Wechselstrom mit verschiedenen Verbrauchern}
	\subsubsection*{Widerstand}
	%TODO Satz das das hgicht tech plastik/metall ding verlustfrei ist
	In \cref{WiderSpannungGleich} und \cref{WiderSpannungWechsel} ist die Spannung über einen Widerstand gegen den Strom aufgetragen. 
	Die Steigung der linearen Fits entspricht dem Widerstand $R$ $\SI{15,72 \pm 0,04}{\Omega}$ bzw. $\SI{15,55 \pm 0,04}{\Omega}$. 
	Der eingestellte Widerstand war \SI{14 \pm 1,7}{\Omega}.

	In \cref{WiderLeistungGleich} und \cref{WiderLeistungWechsel} ist die Leistung gegen das Produkt von Strom und Spannung über den Widerstand aufgetragen. Es ist in beiden Fällen eine Steigung von 1 zu erwarten, da $P = UI$ gilt. Im Fall des Wechsselstroms wurden nur Effektivwerte gemessen, da dies einen Faktor von $\frac{1}{2}$ für die Leistung und 2 Faktoren von $\sqrt{2}$ für $UI$ bedeutet, bleibt die Steigung dieselbe.
	\begin{figure}[tb]
		\includegraphics[width=1\textwidth]{WiderSpannungGleich}
		\centering
		\caption{Die gemessene Gleichspannung über einen Widerstand ist gegen den Gleichstrom aufgetragen.}
		\label{WiderSpannungGleich}
		\centering
	\end{figure}
	\begin{figure}[tb]
		\includegraphics[width=1\textwidth]{WiderSpannungWechsel}
		\centering
		\caption{Die gemessene effektive Wechselspannung über einen Widerstand ist gegen den effektiven Wechselstrom aufgetragen.}
		\label{WiderSpannungWechsel}
		\centering
	\end{figure}

	\begin{figure}[tb]
		\includegraphics[width=1\textwidth]{WiderLeistungGleich}
		\centering
		\caption{Die gemessene Leistung ist gegen das Produkt aus Gleichstrom und Gleichspannung über einen Widerstand aufgetragen.} %TODO ist des effektiv oder net?
		\label{WiderLeistungGleich}
		\centering
	\end{figure}
	
	\begin{figure}[tb]
		\includegraphics[width=1\textwidth]{WiderLeistungWechsel}
		\centering
		\caption{Die gemessene effektive Leistung ist gegen das Produkt aus effektivem Wechselstrom und effektiver Wechselspannung über einen Widerstand aufgetragen.}
		\label{WiderLeistungWechsel}
		\centering
	\end{figure}

	\subsubsection*{Spule}
	Die gemessene effektive Wechselspannung über die Spule ist gegen den effektiven Strom in \cref{SpuleSpannungWechsel} aufgetragen.
	Die Steigung des linearen Fits ist der Scheinwiderstand $|Z|$ $\SI{30,00 \pm 0,06}{\Omega}$. 
	
	Der Widerstand der Spule findet sich in \cref{SpuleSpannungGleich} wieder. 
	In diesem Graphen wurde die Gleichspannung über die Spule gegen den Gleichstrom aufgetragen und analog ist die Steigung des Fits der Widerstand $R_\text{i}$ $\SI{24,04 \pm 0,06}{\Omega}$.

	Aus der Theorie ist folgender Zusammenhang bekannt:
	\begin{equation}
		\bar{P} = U_\text{eff} I_\text{eff} \cos(\phi)
	\end{equation}
	\cref{SpuleLeistungWechsel} beinhaltet die Messwerte für die effektive Leistung in Abhängigket von dem Produkt der effektiven Spannung und des effektiven Stroms. 
	Der linearer Fit hat die Steigung $\SI{0,7965\pm 0,0041}{}$, was $\cos(\phi)$ entsprechen sollte. 
	Es folgt also ein $\phi$ von $\ang{37,202\pm 0,385}$.

	Die Indukivität der Spule lässt sich durch die bereits bestimmten Werte und \cref{Spule} bestimmen.
	\begin{gather}
		\label{Spule}
		|Z| = \sqrt{R_\text{W}^2 + (\omega L)^2} \\
		L = \frac{1}{\omega}\sqrt{|Z|^2-R_\text{W}^2} \\
		R_\text{W} = |Z| \cos(\phi)  \\
		L = \frac{1}{\omega} \sin{\phi} |Z|
	\end{gather}
		Das Stromnetz hat eine Frequenz von \SI{50}{Hz}. %TODO Wirkwiderstand diskussion ist größer weil keine SChwingugn und Selbstindktion
		Es ergibt sich ein Wirkwiderstand von \SI{23,90 \pm 0,06}{\Omega}
		Daraus folgt eine Induktivität von \SI{0,0577 \pm 0,00019}{H}. %TODO Unsicherheit Kette kombi
	

	\begin{figure}[tb]
		\includegraphics[width=1\textwidth]{SpuleSpannungWechsel}
		\centering
		\caption{Die gemessene effektive Wechselspannung über eine Spule ist gegen den effektiver Wechselstrom aufgetragen.}
		\label{SpuleSpannungWechsel}
		\centering
	\end{figure}
	\begin{figure}[tb]
		\includegraphics[width=1\textwidth]{SpuleLeistungWechsel}
		\centering
		\caption{Die gemessene effektive Leistung ist gegen das Produkt aus Wechselstrom und Wechselspannung über eine Spule aufgetragen.}
		\label{SpuleLeistungWechsel}
		\centering
	\end{figure}
	\begin{figure}[tb]
		\includegraphics[width=1\textwidth]{SpuleSpannungGleich}
		\centering
		\caption{Die gemessene Spannung über eine Spule ist gegen den Strom aufgetragen.}
		\label{SpuleSpannungGleich}
		\centering
	\end{figure}

	\subsubsection*{Spule und Kondensator}
	Die gemessene effektive Wechselspannung über die Spule und den Kondensator ist gegen den effektiven Strom in \cref{KondensatorSpannungWechsel} aufgetragen.
	Die Steigung des linearen Fits ist der Betrag des Wechselstromwiderstandes $|Z|$ $\SI{41,81 \pm 0,07}{\Omega}$. 

	Aus der Theorie ist folgender Zusammenhang bekannt:
	\begin{equation}
		\bar{P} = U_\text{eff} I_\text{eff} \cos(\phi)
	\end{equation}
	\cref{KondensatorLeistungWechsel} beinhaltet die Messwerte für die effektive Leistung in Abhängigket von dem Produkt der effektiven Spannung und des effektiven Stroms. 
	Der linearer Fit hat die Steigung $\SI{0,651 \pm 0,004}{}$, was $\cos(\phi)$ entsprechen sollte. 
	Es folgt also ein $\phi$ von $-\ang{49,38 \pm 0,30}$.

	Die Kapaziät lässt sich mittels \cref{Kondensator} bestimmen.
	\begin{gather}
		\label{Kondensator}
		|Z| = \sqrt{R^2 + (\omega L - \frac{1}{\omega C})^2} \\
		C = \frac{1}{\omega^2 L- \omega\sqrt{Z^2-R^2}} \\
		C = \frac{1}{\omega^2 L- \omega|Z|\sin{\phi}} 
	\end{gather}
	Durch Einsetzten ergibt sich eine Kapazität $C$ von \SI{63,8 \pm 2,1}{\mu F}. Auf dem Kondensator war eine Kapazität von \SI{60}{\mu F} angegeben.




	\begin{figure}[tb]
		\includegraphics[width=1\textwidth]{KondensatorSpannungWechsel}
		\centering
		\caption{Die gemessene Wechselspannung über eine Spule und einen Kondensator ist gegen den Wechselstrom aufgetragen.}
		\label{KondesatorSpannungWechsel}
		\centering
	\end{figure}
	\begin{figure}[tb]
		\includegraphics[width=1\textwidth]{KondensatorLeistungWechsel}
		\centering
		\caption{Die gemessene effektive Leistung ist gegen das Produkt aus Wechselstrom und Wechselspannung über eine Spule und einen Kondensator aufgetragen.}
		\label{KondensatorLeistungWechsel}
		\centering
	\end{figure}
	



	




	\subsection{Diskussion}
	%TODO Bezug/Nutzten oder sonst was
	%TODO auch hier die Hypothese wiederholen
	
	\section{Schlussfolgerung}
	%TODO Rückgriff auf Hypothese und drittes Nennen dieser
	Im Fall der Innenwiderstände von Akkumulatorzellen konnte festgestellt werden
	Bei der Verlustleistung verschiedener Verbraucher 
	%TODO Quellen zitieren, Websiten mit Zugriffsdatum
	%TODO Verweise auf das Laborbuch (sind erlaubt)
	%TODO Tabelle + Bilder mit Beschriftung
	%\printbibliography
\end{document}
