% Autor: Leonhard Segger, Alexander Neuwirth
% Datum: 2017-10-30
\documentclass[
	% Papierformat
	a4paper,
	% Schriftgröße (beliebige Größen mit „fontsize=Xpt“)
	12pt,
	% Schreibt die Papiergröße korrekt ins Ausgabedokument
	pagesize,
	% Sprache für z.B. Babel
	ngerman
]{scrartcl}

% Achtung: Die Reihenfolge der Pakete kann (leider) wichtig sein!
% Insbesondere sollten (so wie hier) babel, fontenc und inputenc (in dieser
% Reihenfolge) als Erstes und hyperref und cleveref (Reihenfolge auch hier
% beachten) als Letztes geladen werden!

% Silbentrennung etc.; Sprache wird durch Option bei \documentclass festgelegt
\usepackage{babel}
% Verwendung der Zeichentabelle T1 (Sonderzeichen etc.)
\usepackage[T1]{fontenc}
% Legt die Zeichenkodierung der Eingabedatei fest, z.B. UTF-8
\usepackage[utf8]{inputenc}
% Schriftart
\usepackage{lmodern}
% Zusätzliche Sonderzeichen
\usepackage{textcomp}

% Mathepaket (intlimits: Grenzen über/unter Integralzeichen)
\usepackage[intlimits]{amsmath}
% Ermöglicht die Nutzung von \SI{Zahl}{Einheit} u.a.
\usepackage{siunitx}
% Zum flexiblen Einbinden von Grafiken (\includegraphics)
\usepackage{graphicx}
% Abbildungen im Fließtext
\usepackage{wrapfig}
% Abbildungen nebeneinander (subfigure, subtable)
\usepackage{subcaption}
% Funktionen für Anführungszeichen
\usepackage{csquotes}
% Zitieren, Bibliographie
\usepackage{biblatex}

% Verlinkt Textstellen im PDF-Dokument
\usepackage[unicode]{hyperref}
% "Schlaue" Referenzen (nach hyperref laden!)
\usepackage{cleveref}
% Zur Darstellung von Webadressen
\usepackage{url}
%chemische Formeln
\usepackage[version=4]{mhchem}
% siunitx: Deutsche Ausgabe, Messfehler getrennt mit ± ausgeben
\usepackage{floatrow}
\floatsetup[table]{capposition=top}
\sisetup{
	locale=DE,
	separate-uncertainty
}
%\bibliography{6Mi_S2_25-10-2017_References}

\begin{document}
	
	\begin{titlepage}
		\centering
		{\scshape\LARGE Versuchsbericht zu \par}
		\vspace{1cm}
		{\scshape\huge M1 - Drehpendel nach Pohl\par}
		\vspace{2.5cm}
		{\LARGE Gruppe 6Mi \par}
		\vspace{0.5cm}
		
		{\large Alexander Neuwirth (E-Mail: a\_neuw01@wwu.de) \par}
		{\large Leonhard Segger (E-Mail: l\_segg03@uni-muenster.de) \par}
		\vfill
		
		durchgeführt am 21.11.2017\par
		betreut von\par
		{\large Torsten Stiehm}
		
		\vfill
		
		{\large \today\par}
	\end{titlepage}
	\tableofcontents
	\newpage
	
	\section{Kurzfassung}
	Das Drehpendel nach Pohl ist ein Beispiel für schwingfähige Systeme. Es kann mit einem Exzenter eine erzwungene Oszillation eingestellt werden und mithilfe einer Wirbelstrombremse eine Dämpfung erzeugt werden. Dies haben wir im Folgenden gemacht und Dämpfungen in Abhängigkeit von dem Stromfluss durch die Wirbelstrombremse bestimmt. Dann wurde der Zusammenhang zwischen Dämpfung und Resonanzfrequenz untersucht und ein Vergleich zum Fadenpendel aufgestellt.
	Zuletzt wurde noch das Verhalten des Drehpendels unter Einfluss einer Nichtlinearität untersucht.
	\section{Methoden}
	
	\subsection{Drehpendel ohne Dämpfung}%...
	Zunächst haben wir die Eigenfrequenz, bzw. Periodendauer, des Drehpendels (annähernd) ohne Dämpfung mithilfe einer Stoppuhr bestimmt. Um den Fehler, der durch die menschliche Reaktionszeit entsteht, zu minimieren, haben wir die Zeit, die das Drehpendel für 20 Schwingungen benötigt, gemessen, um dann über diese zu mitteln.
	Als Anfangs- und Endpunkt der Messung haben wir die Ruhelage der Scheibe gewählt, da sie sich dort näherungsweise mit konstanter Geschwindigkeit bewegt und man somit den Reaktionsfehler leichter ausgleichen kann.
	Alternativ hätte man den linken oder rechten Wendepunkt der Bewegung wählen können. An dieser Stelle bewegt sich das Pendel allerdings langsam und der Zeitraum, in dem das Pendel stillsteht, ist verhältnismässig groß, weshalb das Ende der Bewegung kaum exakt zu erkennen ist. %Von sich selbst darf man ja wohl plagiarisieren.
	Dann haben wir dieselbe Messung erneut durchgeführt. Diesmal wurde die Drehung digital mit einem Messrad, welches durch einem Faden mit dem Schwingrad verbunden war, erfasst und als Messkurve mit dem Computer aufgezeichnet.%Cyber Cyber
	\subsection{Drehpendel mit Dämpfung}
	Im Anschluss daran fügten wir eine Dämpfung in Form einer Wirbelstrombremse hinzu. Von dem sich ergebenden Schwingungsvorgang haben wir drei Messkurven mit jeweils unterschiedlichen Dämpfungen, also unterschiedlichem Stromfluss durch die Spulen der Wirbelstrombremse, aufgenommen. Hieraus konnten wir jeweils die Eigenfrequenz und die Dämpfung aus der Abnahme der Amplitude bestimmen. %Sollte man hier die konkreten Werte des Stromflusses nennen? Denke nicht.
	\subsection{Drehpendel mit Exzenter und Dämpfung} 
	Nun haben wir das Drehpendel mit einem Exzenter (in Form eines Motors, der die Spiralfeder anregt) und der Wirbelstrombremse betrieben. Hierbei haben wir darauf geachtet die Messung erst zu starten, nachdem das Pendel die Einschwingphase überwunden hatte, also sich Frequenz und Amplitude nicht mehr merklich änderten.
	Zunächst nahmen wir eine Kalibrierkurve auf, um den Zusammenhang zwischen der Frequenz der Anregung $ f $ und der Spannung $ U $ am Tacho-Ausgang des Motors zu quantifizieren.
	Anschließend haben wir für drei verschiedene Dämpfungen je 20 Messungen mit unterschiedlicher Anregungsfrequenz $ f $ durchgeführt. Gemessen wurde jeweils für ca. \SI{20}{\second}. 
	\subsection{Qualitative Beobachtungen und Vergleich zum Fadenpendel}
	Wir haben bei kleiner Dämpfung den Phasenunterschied zwischen Anregung und Drehpendel bei Frequenzen über, unter und nahe bei der Resonanzfrequenz beobachtet. Dann haben wir dieselbe Untersuchung bei einem Fadenpendel durchgeführt, um die Parallelen erkennen zu können.
	\subsection{Einführung einer Nichtlinearität}
	Zuletzt wurde noch eine Nichtlinearität in Form eines kleinen Gewichtes eingeführt. Diese Gewicht haben wir am oberen Rand der Scheibe (eine Daumenbreite neben dem obersten Punkt) in Ruhelage befestigt. Dann haben wir das Verhalten des Drehpendels bei steigender sowie bei sinkender Anregungsfrequenz beobachtet. Anschließend haben wir näherungsweise die Resonanzfrequenz eingestellt und die Schwingung beobachtet.
	
	\section{Ergebnisse und Diskussion}
	
	\subsection{Drehpendel ohne Dämpfung}
	Die Messung mit einer Stoppuhr ergibt eine Dauer von $T_{20} =\SI{28,15}{\second}$ für 20 Schwingungen, also pro Schwingung $ T_\text{Uhr} = \SI{1,41}{\second}$. %TODO seltsame Underfull hbox Warnung
	Für die Unsicherheit von \(T\) gilt: \\
	\begin{itemize}
		\item  Digitalanzeige der Stoppuhr \SI{\pm 0,005}{\second}, (Typ B): \( u_B(T) = \frac{2\cdot \SI{0,005}{\second}}{2 \cdot 20 \sqrt{3}} = \SI{0,00014}{\second} \)
		\item Reaktionszeit \SI{\pm 0,1}{\second}, (Typ B): \( u_B(T) \approx \frac{2\cdot \SI{0,1}{\second}}{2 \cdot 20 \sqrt{3}} = 0.0029\si{s} \)
		\item Komb. Unsicherheit: $ u_\text{Uhr}(T) = \sqrt{(0,00014\si{s})^2+(0.0029\si{s})^2} \approx 0,0029 \si{s} $
	\end{itemize} 
	$ \implies T_\text{Uhr} = \SI{1,41 \pm 0,0029}{\second} $ \newline 
	Für die Frequenz ergibt sich daraus $ f_\text{Uhr} = \frac{1}{T_\text{Uhr}} = \SI{0,71}{\per \second}$ und für die Unsicherheit: \newline
	
	\begin{equation*}
		u_\text{Uhr}(f) = \left|  \frac{\partial f}{\partial T_\text{Uhr}} u_\text{Uhr}(T_\text{Uhr}) \right| = \left| - \frac{1}{T_\text{Uhr}^2}u_\text{Uhr}(T_\text{Uhr}) \right| \approx \SI{0,0015}{\per \second} %Ich hoffe mal das ist rechnerisch richtig, man könnte sich beschweren, dass wir bei der Unsicherheit mehr Stellen von f,t,u angeben müssten
	\end{equation*}
	Insgesamt: $f_\text{Uhr} = \SI{0,709 \pm 0,0015}{\per \second}  $ \\
	\begin{figure}[htb]
		\includegraphics[width=1\textwidth]{Ungedaempfte_Schwingung_Graph}
		\centering
		\caption{Das Ergebnis der mit dem Computer aufgezeichneten Messkurve für die ungedämpfte Schwingung}
		\label{ungedämpfte_Schwingung}
		\centering
	\end{figure}
	In \cref{ungedämpfte_Schwingung} wurden die Messergebnisse der (näherungsweise) ungedämpften Schwingung aufgetragen. Dann wurde ein Fit gemäß dem \enquote{Scaled Levenberg-Marquardt algorithm} mit der zugrunde liegenden Funktion
	\begin{equation}
	y(t) = a \sin ((x-b)2 \pi f) \exp^{-x \cdot b} + c
	\end{equation}
	
	durchgeführt. Dabei ist $a$ die Amplitude bei $t=0$, $b$ die Phase bei $t=0$  (beides nicht weiter interessant, weil nur abhängig vom Beginn der Messung), $f$ die Frequenz und $d$ die Dämpfung.\\
	Zusätzlich entsteht eine Unsicherheit durch ein mögliches Mitrutschen des Seils an beiden Scheiben, den wir im Folgenden großzügig mit 
	\begin{equation*}
		u(y) = \frac{2 \cdot \SI{0,01}{\meter}}{2 \sqrt{3}} \approx \SI{0,0058}{\meter}
	\end{equation*}
	abschätzen. Da die Bewegung allerdings gleichermaßen in beide Richtungen stattfindet, wird dieser Fehler nicht mit steigender Messungsdauer größer.
	Wir gehen davon aus, dass der unbekannte Fehler des Messgerätes bei Messung der Auslenkung gegenüber jenem verschwindet. Gleiches nehmen wir für die Messung der Zeit an, die mithilfe des Computers exakt genug erfasst werden kann, dass der Einfluss dieses Fehlers zu vernachlässigen ist. \par
	Der Fit ergibt die Frequenz $ f = \SI{0,67308 \pm 0,00004}{\per \second} $.
	Außerdem lässt sich anhand des Diagramms erkennen (und an der aus dem Fit folgenden Dämpfung $ d= \SI{0,05}{\per \second} $), dass die Annahme, die Schwingung sei ungedämpft, recht unpräzise ist. \\
	Wenn man das Ergebnis der manuellen Messung ($f_\text{Uhr} = \SI{0,71 \pm 0,0015}{\per \second}$) mit dem der mit dem Computer aufgezeichneten Messung ($ f = \SI{0,67308 \pm 0,00004}{\per \second} $) vergleicht, stellt man fest, dass die Ergebnisse zwar nah beieinander, aber die Differenz nicht innerhalb der Fehler liegt.
	Dies ist auf mangelnde exakte Kontrolle der Versuchsrahmenbedingungen (die sich von der manuellen zur automatisierten Messung geändert haben) und den unbekannte Fehler des Messgeräts zurückzuführen sind.
	
	\subsection{Drehpendel mit Dämpfung}
	
	\begin{figure}[htb]
		\includegraphics[width=1\textwidth]{Gedaempfte_Schwingung_Graph}
		\centering
		\caption{Das Ergebnis der mit dem Computer aufgezeichneten Messkurve für die gedämpfte Schwingung bei Betrieb der Wirbelstrombremse mit \SI{0,2}{\ampere}, \SI{0,3}{\ampere} und \SI{0,5}{\ampere}.}
		\label{gedämpfte_Schwingung}
		\centering
	\end{figure}
	Eine Anwendung desselben Fits auf die Messungen mit einer Wirbelstrombremse, durch die ein Strom $ I $ fließt, liefert folgendes Ergebnis in \cref{gedaempftTab} für die Frequenz $ f $ und die Dämpfung $ d $.\\
	\begin{table}[htb]
		\centering
		\begin{tabular}{ r | c | c | c |}
			$I /\si{\ampere}$& 0,2 & 0,3 & 0,5\\ \hline
			$f  /\si{\per \second}$ & \num{0,6717 \pm 0,0001} & \num{0,6728 \pm 0,0002} & \num{0,6741 \pm 0,0008}\\
			$d /\si{\per \meter}  $ & \num{0,108 \pm 0,001} & \num{0,159 \pm 0,001} & \num{0,337 \pm 0,005}
		\end{tabular}
		\caption{Ergebnisse des Fits für die Frequenz $ f $ und die Dämpfung $ d $ bei unterschiedlichen Stromstärken der Wirbelstrombremse}
		\label{gedaempftTab}
	\end{table}
	Man erkennt, dass sich die Frequenzen in Abhängigkeit von der Dämpfung kaum verändern. Dies entspricht dem Wissen aus der Versuchsanleitung, welche gemäß \cref{Frequenzverschiebung} eine Differenz zwischen $ \omega_0 $ und $ \tilde{\omega_0} $ von nur wenigen Promille in der Praxis voraussagt.
	\begin{equation}
		\tilde{\omega_0} = \sqrt{\omega_0^2 - d^2}
		\label{Frequenzverschiebung}
	\end{equation}
	\subsection{Drehpendel mit Exzenter und Dämpfung}
	
	Um zunächst die Kalibrierkurve für den Zusammenhang zwischen Spannung am Tacho-Ausgang des Motors und Anregungsfrequenz aufzustellen, wurde ein Fit gemäß dem \enquote{Scaled Levenberg-Marquardt algorithm} mit der zugrunde liegenden Funktion
	\begin{equation}
	y(t) = a \sin ((x-b)2 \pi f) + c
	\end{equation}
	für jede der angelegten Spannungen durchgeführt. Aus diesen Fits ergeben sich Fehler, die alle kleiner sind als $ u_x = 0,002 $. Dann wurden in \cref{Kalibirierkurve} die Frequenzen gegen die Spannung aufgetragen. Die Unsicherheit des Multimeters, mit dem die Spannung gemessen wurde, wird im Folgenden betrachtet: \\
	Der Hersteller des Multimeters gibt eine Messtoleranz von ±0,5\% des abgelesenen Wertes für Gleichspannungsmessungen in den Messbereichen 2 V und 20 V an. Dies ist eine Typ B Unsicherheit mit rechteckiger WDF. Außerdem lässt sich die Spannung nur bis auf zwei Nachkommastellen ablesen, wodurch eine zusätzliche Ungenauigkeit von $\pm 0,005\si{V}$ entsteht. Dies ist ebenfalls eine Typ B Unsicherheit mit rechteckiger WDF. Insgesamt ergibt sich: \\
	$u_1= \frac{0,01}{2\sqrt{3}} \si{V} \approx 0,0029 \si{V}$ \\
	$u_2= \frac{2 \cdot 0,005 \cdot \text{Messwert}}{2 \sqrt{3}}$ \\ %kp ob das schön ist
	Für Messwerte kleiner \SI{20}{\volt} schätzen wir die Unsicherheit nach oben ab mit:
	\begin{equation*}
		\sqrt{(\frac{0,01 \cdot \text{Messwert}}{2 \sqrt{3}})^2 + 0,0029^2} \si{V} <  \SI{0,06}{\volt}
	\end{equation*}
	
	\begin{figure}[htb]
		\includegraphics[width=1\textwidth]{Kalibrierkurve_Graph}
		\centering
		\caption{Hier ist die Abhängigkeit zwischen Spannung und Anregungsfrequenz dargestellt.}
		\label{Kalibirierkurve}
		\centering
	\end{figure}
	Aus \cref{Kalibirierkurve} lässt sich ein linearer Zusammenhang zwischen Spannung und Frequenz erkennen, der durch den Punkt (0,0) laufen muss, da der Motor ohne Spannung sich nicht bewegen kann. Wenn man einen dementsprechenden Fit durchführt, ergibt sich ein Zusammenhang von $ f_0 = U \cdot \SI{0,073 \pm 0,001}{\per \second \per \volt} $. Mit diesem Wissen lassen sich nun die Ergebnisse für die Amplitude in Abhängigkeit von der Anregungsfrequenz auswerten.
	\par
	Dazu wurde für jede der drei Dämpfungen und jede der jeweils 20 Anregungsfrequenzen die Auslenkung gegen die Zeit aufgetragen. Dann wurde mit dem Auge die Amplitude abgelesen. Ein präziseres Ergebnis hätte sich ergeben, wenn man für jede Periode dieser Schwingungen die Amplitude abgelesen hätte, aber dies hätte ein manuelles Ablesen von $3 \cdot 20 \cdot 20 = 1200 $ Amplituden benötigt, was zeitlich nicht umzusetzen war. Ebenso hätte man jeweils einen Fit anwenden können, aber dieser gab ohne starke manuelle Anpassungen kein zufriedenstellendes Ergebnis.
	Als Ablesefehler nehmen wir hier $\pm \SI{0,001}{\meter}$ an, also $ u(\varphi) = \frac{\SI{0,002}{\meter}}{2\sqrt{3}} \approx \SI{0,00058}{\meter} $. Diesem Fehler gegenüber verschwindet der x-Fehler, der durch die Kalibrierkurve und die Ungenauigkeit des Messgeräts entsteht. Der y-Fehler ist klein genug, um beim Ablesen des Maximums kaum eine Rolle zu spielen.
	\begin{figure}[H]
		\includegraphics[width=1\textwidth]{erzwungene_Schwingung_0,2A}
		\centering
		\caption{Die maximale Amplitude $ \varphi_0 $ in Abhängigkeit von der Frequenz der Anregung $ f_0 $ bei $I=\SI{0,2}{\ampere}$}
		\label{erzw02A_Schwingung}
		\centering
	\end{figure}
	\begin{figure}[H]
		\includegraphics[width=1\textwidth]{erzwungene_Schwingung_0,4A}
		\centering
		\caption{Die maximale Amplitude $ \varphi_0 $ in Abhängigkeit von der Frequenz der Anregung $ f_0 $  bei $I=\SI{0,4}{\ampere}$}
		\label{erzw04A_Schwingung}
		\centering
	\end{figure}
	\begin{figure}[H]
		\includegraphics[width=1\textwidth]{erzwungene_Schwingung_0,6A}
		\centering
		\caption{Die maximale Amplitude $ \varphi_0 $ in Abhängigkeit von der Frequenz der Anregung $ f_0 $  bei $I=\SI{0,6}{\ampere}$}
		\label{erzw06A_Schwingung}
		\centering
	\end{figure}
	Aus \cref{erzw02A_Schwingung}, \cref{erzw04A_Schwingung} und \cref{erzw06A_Schwingung} lässt sich die Resonanzfrequenz des Drehpendels bei den jeweiligen Dämpfungen anhand des Hochpunktes ablesen. Das Ergebnis ist in \cref{erzwTab} zu sehen. Dabei ist die Unsicherheit bei einem Ablesefehler von \SI{\pm 0,02}{\per \second} $u(f_R)=\frac{0,04}{2\sqrt{6}} \approx \SI{0,008}{\per \second}$. 
	\begin{table}[htb]
		\centering
		\begin{tabular}{ r | c | c | c |}
			Strom $I /\si{\ampere}$& 0,2 & 0,4 & 0,6\\ \hline
			Resonanzfrequenz $ f_R / \si{\per \second}$& \num{0,70 \pm 0,008} & \num{0,68 \pm 0,008} & \num{0,68 \pm 0,008}
		\end{tabular}
		\caption{Ergebnisse für die Resonanzfrequenz $ f_R $ bei unterschiedlichen Stromstärken der Wirbelstrombremse}
		\label{erzwTab}
	\end{table}
	Hier wäre zu erwarten, dass keine zwei Resonanzfrequenzen gleich sind. Dass dies doch so ist, liegt an der ungenauen Ablesbarkeit des Maximums für $I = \SI{0,6}{\ampere}$.
	
	\subsection{Qualitative Beobachtungen und Vergleich zum Fadenpendel}
	Die Beobachtung der Phasenbeziehung zwischen Anregung und Drehpendel ergibt, dass die Phasendifferenz bei hohen Frequenzen ungefähr $ \pi $  beträgt, bei kleinen keine Phasenverschiebung erkennbar ist und bei Frequenzen nahe der Eigenfrequenz das Pendel der Anregung um ungefähr $ \frac{\pi}{2} $ hinterher hängt. Beim manuellen Pendeln eines Fadenpendels entspricht die Frequenz, die man dem Pendel durch die Bewegung der Hand vorgibt, der Anregungsfrequenz und die Frequenz, die sich aus Ortsfaktor und Länge des Fadens ergibt, der Eigenfrequenz. Auch hier ist im Fall der Resonanzfrequenz eine Phasenverschiebung von $  \frac{\pi}{2} $ zu beobachten. Eine deutlich tiefere Frequenz zu erzwingen ist schwierig, da dann die Trägheit des Pendels überwiegt und bei kleinen Frequenzen keine Schwingung, sondern nur eine vertikale Bewegung, stattfindet. Bei hohen Frequenzen lässt sich allerdings tatsächlich in ähnlicher Weise eine Phasenverschiebung um $ \pi $ beobachten.
	\subsection{Einführung einer Nichtlinearität}
	Die Einführung einer Nichtlinearität sorgt für ein deutlich weniger gleichmäßiges Verhalten des Pendels. Eine langsame Erhöhung der Anregungsfrequenz sorgt dafür, dass die Bewegung kurz stockt, sobald die anregende Frequenz die Eigenfrequenz überschreitet. Eine langsame Verringerung der Anregungsfrequenz sorgt dafür, dass die Schwingung zunächst nur auf der Seite des Gewichts stattfindet. Nahe der Eigenfrequenz schlägt die Schwingung dann auch auf die andere Seite über, um dann wieder auf die Seite des Gewichts zurückzukehren. Bei näherungsweisem Einstellen der Resonanzfrequenz ist der Ausschlag maximal und die Dämpfung musste erhöht werden, damit das Pendel nicht an das versuchsaufbaubedingte Maximum der Auslenkung (Begrenzer, die einen Überschlag der Scheibe verhindern) zu stoßen. Wenn man dies tut, kann man dieselbe Phasenverschiebung von $ \pm \frac{\pi}{2} $ wie zuvor beobachten.
	\section{Schlussfolgerung}
	
	Insgesamt gesehen konnten mehrere Zusammenhänge beobachtet werden. Zunächst konnte die Frequenz einer ungedämpften Schwingung ohne Exzenter bestimmt werden. Hierbei blieb eine leichte Diskrepanz zwischen automatisierter Messung und manueller Messung mit einer Stoppuhr. Da diese Untersuchung allerdings ohnehin nur eine Aussage über das exakte Drehpendel, das für den Versuch benutzt wurde, macht, ist die Tragweite dieses Ergebnisses eher gering. Ähnliches gilt für die Untersuchung der Dämpfung in Abhängigkeit von dem Stromfluss durch die Wirbelstrombremse. Deutlich mehr Messungen hätten es erlaubt die tatsächliche Abhängigkeit der Dämpfung von der Stärke des Magnetfeldes in der Scheibe zu bestimmen. Dies hätte auch die Untersuchung der Resonanzfrequenz auf andere Fälle übertragbar gemacht, da man sie nicht mehr in Abhängigkeit vom Stromfluss durch die Spulen hätte angeben müssen, sondern in Abhängigkeit von der Dämpfung.
	Die qualitativen Ergebnisse in Bezug auf die Phasenverschiebung wären deutlich aussagekräftiger gewesen, hätte man die Phase am Exzenter messen können und sich nicht auf die für solche Fälle recht schlecht geeignete menschliche Wahrnehmung verlassen müssen. Im Falle der Nichtlinearität wäre wohl selbst eine solche genauere Untersuchung wenig zielführend gewesen, da sich der konkrete Fall eines Gewichtes dieser Form an exakt der von uns gewählten Stelle schlecht auf andere Situationen übertragen lässt.
	%\printbibliography
\end{document}
