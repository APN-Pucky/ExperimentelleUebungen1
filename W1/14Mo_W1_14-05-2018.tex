% Autor: Leonhard Segger, Alexander Neuwirth
% Datum: 2017-10-30
\documentclass[
	% Papierformat
	a4paper,
	% Schriftgröße (beliebige Größen mit „fontsize=Xpt“)
	12pt,
	% Schreibt die Papiergröße korrekt ins Ausgabedokument
	pagesize,
	% Sprache für z.B. Babel
	ngerman
]{scrartcl}

% Achtung: Die Reihenfolge der Pakete kann (leider) wichtig sein!
% Insbesondere sollten (so wie hier) babel, fontenc und inputenc (in dieser
% Reihenfolge) als Erstes und hyperref und cleveref (Reihenfolge auch hier
% beachten) als Letztes geladen werden!

% Silbentrennung etc.; Sprache wird durch Option bei \documentclass festgelegt
\usepackage{babel}
% Verwendung der Zeichentabelle T1 (Sonderzeichen etc.)
\usepackage[T1]{fontenc}
% Legt die Zeichenkodierung der Eingabedatei fest, z.B. UTF-8
\usepackage[utf8]{inputenc}
% Schriftart
\usepackage{lmodern}
% Zusätzliche Sonderzeichen
\usepackage{textcomp}

% Mathepaket (intlimits: Grenzen über/unter Integralzeichen)
\usepackage[intlimits]{amsmath}
% Ermöglicht die Nutzung von \SI{Zahl}{Einheit} u.a.
\usepackage{siunitx}
% Zum flexiblen Einbinden von Grafiken (\includegraphics)
\usepackage{graphicx}
% Abbildungen im Fließtext
\usepackage{wrapfig}
% Abbildungen nebeneinander (subfigure, subtable)
\usepackage{subcaption}
% Funktionen für Anführungszeichen
\usepackage{csquotes} 
\MakeOuterQuote{"}
% Zitieren, Bibliographie
\usepackage{biblatex}


% Zur Darstellung von Webadressen
\usepackage{url}
%chemische Formeln
\usepackage[version=4]{mhchem}
% siunitx: Deutsche Ausgabe, Messfehler getrennt mit ± ausgeben
\usepackage{floatrow}
\floatsetup[table]{capposition=top}
\usepackage{float}
% Verlinkt Textstellen im PDF-Dokument
\usepackage[unicode]{hyperref}
% "Schlaue" Referenzen (nach hyperref laden!)
\usepackage{cleveref}
\sisetup{
	locale=DE,
	separate-uncertainty
}
%\bibliography{14Mo_W1_14-05-2018_References}

\begin{document}
	
	\begin{titlepage}
		\centering
		{\scshape\LARGE Versuchsbericht zu \par}
		\vspace{1cm}
		{\scshape\huge W1 - Stirling-Motor \par} 
		\vspace{2.5cm}
		{\LARGE Gruppe 14Mo \par}
		\vspace{0.5cm}
		
		{\large Alexander Neuwirth (E-Mail: a\_neuw01@wwu.de) \par}
		{\large Leonhard Segger (E-Mail: l\_segg03@uni-muenster.de) \par}
		\vfill
		
		durchgeführt am 14.05.2018\par 
		betreut von\par
		{\large Torsten Stiehm} 
		
		\vfill
		
		{\large \today\par}
	\end{titlepage}
	\tableofcontents
	\newpage

	\section{Kurzfassung}
	%TODO Hypothese	und deren Ergebnis, wenn Hypothese ist, dass nur Theorie erfüllt, sagen: Erwartung: Theorie aus einführung (mit reflink) erfüllt
	%TODO Ergebnisse, auch Zahlen, mindestens wenn's halbwegs Sinn ergibt
	%TODO Was wurde gemacht
	%TODO manche leute wollen Passiv oder "man", manche nicht
	
	\section{Methoden} \label{sec_Methoden}
	%TODO Bilder von der Website klauen
	
	%TODO Motor frequenz FFT
	%TODO 5 fach millilitter für Durchfluss
	In \cref{Aufbau} ist der verwendete Stirling-Motor dargestellt.
	Zunächst wurden die Reibungsverluste durch Reibung bestimmt.
	Dazu wurde der Motor mit offenem Zylinderkopf betrieben, sodass der Druck im Zylinder konstant bei Luftdruck lag und abgesehen von Reibungswärme keine Heiz- oder Kühlwirkung entstehen konnte.
	Bei diesem Betrieb wurde dann zunächst der Volumendurchsatz des Kühlwassers bestimmt, indem an der Pumpe mit Stoppuhr und Messzylinder die Zeit und der Füllstand nach ca. \SI{9}{s} gemessen.
	Dies wurde fünf mal durchgeführt, um den den Mittelwert bilden und den Messfehler verringern zu können.
	Dann wurde die Temperaturerhöhung des aus dem Motor abfließenden Wassers gegenüber dem Wasser im Tank, der die Pumpe speist, gemessen.
	Für diese und die folgenden Messungen wurde für die Drehzahl des Elektromotors etwa \SI{3}{\kilo \hertz} gewählt.
	Gemessen wurde die Frequenz, indem eine Fouriertransformation durchgeführt wurde und die Spitze betrachtet wurde.
	
	Im zweiten Teil wurde ein Reagenzglas mit ca. \SI{1}{\milli \liter} (mit einer Pipette befüllt) in den Zylinderkopf eingesetzt und dieser geschlossen.
	Dann wurde bei gleicher Umlaufrichtung (als Wärmemaschine) die Temperatur im Reagenzglas in Abhängigkeit von der Zeit gemessen und am Ende wie zuvor der Temperaturunterschied zwischen abfließendem Kühlwasser und Wassertank gemessen.
	
	Sobald ca. \SI{-25}{\degreeCelsius} erreicht waren, wurde die Umlaufrichtung des Elektromotors geändert, sodass der Stirling-Motor als Wärmepumpe fungierte und erneut die Temperatur im Reagenzglas in Abhängigkeit von der Zeit gemessen.
	Auch hier wurde am Ende der Temperaturunterschied des Kühlwassers gemessen.
	
	\begin{figure}[H]
		\includegraphics[width=0.29\textwidth]{Aufbau}
		\centering
		\caption{Aufbau des Stirling-Motors.}
		\label{Aufbau}
		\centering
	\end{figure} 

	\section{Ergebnisse und Diskussion}
	%TODO Unsicherheiten
	

	\subsection{Beobachtung und Datenanalyse}
	
	\subsubsection{Unsicherheiten} 
	Die Unsicherheiten wurden gemäß GUM ermittelt. 
	Außerdem wurde für Unsicherheitsrechnungen die Python Bibliothek "uncertainties" verwendet.
	\begin{description}
		\item[Messzylinder] Die Unsicherheit des Messzylinders wurde mit \SI{0,04}{mL} abgeschätzt (dreieckige WDF).
		\item[Stoppuhr] Die Stoppuhr zeigt Sekunden mit Zwei Nachkommastellen an, woraus eine Unsicherheit von \SI{0,004}{s} folgt (rechteckige WDF), jedoch hat die Reaktionszeit einen größeren Einfluss, wesshalb eine Unsicherheit von \SI{0,1}{s} angenommen wird. 
		\item[Pipette] Auf der Pipette, die zum Füllen des Reagenzglases m Zylinderkopf verwendet wurde, ist eine Unscherheit von \SI{0,007}{mL} angegeben.
		\item[Thermometer] Die Unsicherheit des Kühlwasserthermometers vom Typ K ist \SI{1,5}{\degreeCelsius} in dem gemessenen Temperaturinterval. Da diese für das Messen von Temperaturdifferenzen kaum Einfluss hat, werden die Unsichereiten aufgrund der Schwankungen mit \SI{0,05}{\degreeCelsius} abgeschätzt.
		\item[Motorfrequenz] Die Unsicherheit der, durch FFT ermittelten, Frequenzen wurde mit \SI{0,01}{Hz} abgeschätzt, da die Frequenz kaum schwankte und keine anderen Frequenzen im FFT auftraten.
	
	\end{description}
	\subsubsection{Bestimmung der Reibungsverluste}
	Die Reibungsverluste lassen sich aus der Erwärmung des Kühlwassers beim Betrieb der Wärmepumpe bzw. Kältemaschine bei offenem Zylinderkopf bestimmen.
	Die zugeführte Wärmemenge $\Delta{Q}$ ist proportional zur Temperaturänderung $\Delta{T}$:
	\begin{equation}
	\Delta{Q} = C_W \cdot \Delta{T} = c \cdot m \cdot \Delta{T}
	\label{eq_Wärme}
	\end{equation} 
	Für Wasser beträgt die spezifische Wärme $c_{H_2O}$ = \SI{4,185}{J/g/K}.
	Die Masse $m$ im System ist nicht direkt bestimmbar, der Durchfluss des Kühlwasser $d=m/t$ hingegen schon. %TODO Besserer Satz?
	Somit lässt sich mit \cref{eq_Wärme} die an das Kühlwasser abgegebenen Leistung $\Delta{Q}/t$ ermitteln.
	Die gesuchte Reibungsarbeit pro Umlauf erhält man durch Division der Leistung durch die Frequenz des Motors.
	Es folgt:
	\begin{equation}
	W_R = c_{H_2O} \cdot \frac{d}{f} \cdot \Delta{T}
	\label{eq_Reibungsarbeit}
	\end{equation}
	Der Durchfluss d ergibt sich indem man die geflossene Wassermenge $v$ durch die gestoppte Zeit $t$ dividiert und mit der Dichte $\rho_{H_2O}$ multipliziert. Aus \cref{tab_Durchfluss} ergibt sich ein Mittelwert von \SI{4,61+-0,13}{mL/s} und somit ein Durchfluss d = \SI{4,61+-0,13}{g/s}.
	
	Die Frequenz des Motors wurde mittels FFT auf \SI{3,15+-0,01}{Hz} eingestellt (vgl. \cref{sec_Methoden}).
	Die Temperaturänderung des Kühlwassers $\Delta{T}$ betrug \SI{0,5+-0,05}{\degreeCelsius}.
	Es folgt eine Reibungsarbeit pro Umlauf gemäß \cref{eq_Reibungsarbeit} von $W_R=\SI{2,76+-0,29}{J}$.
	\begin{table}[H]
		\centering
		\begin{tabular}{ c | c }
			Wassermenge $v$ in \SI{}{mL} & Zeit $t$ in \SI{}{s} \\ \hline
			\SI{38,0+-0,03}{}&\SI{8,16+-0,1}{}\\
			\SI{40,8+-0,03}{}&\SI{8,50+-0,1}{}\\
			\SI{41,4+-0,03}{}&\SI{9,34+-0,1}{}\\
			\SI{49,2+-0,03}{}&\SI{10,78+-0,1}{}\\
			\SI{47,0+-0,03}{}&\SI{10,22+-0,1}{}\\
		\end{tabular}
		\caption{Gemessene Kühlwassermenge die durch den Striling-Motor in einer bestimmten Zeit fließt.}
		\label{tab_Durchfluss} 
	\end{table}
	
	\subsubsection{Bestimmung der Kühlleistung} \label{sssec_Kühlleistung}
	Die pro Umlauf aufzuwendende mechanische Arbeit $-W$ ist durch
	\begin{equation}
	W =  Q_1 - Q_2 + W_R
	\end{equation}
	gegeben.
	Die Wärme $Q_2$ wird dem Zylinderkopf pro Umlauf entzogen und die Wärme $-Q_1$ wird dem Kühlwasser zu geführt.
	
	$Q_1+W_R$ lässt sich analog zur Reibungsarbeit pro Umlauf mit \cref{eq_Reibungsarbeit} bestimmen.	
	Die gemessene Temperaturänderung des Kühlwassers betrug $\Delta{T}$=$\SI{1,0+-0,05}{\degreeCelsius}$ woraus $Q_1+W_R$=$\SI{5,53+-0,33}{J}$ folgt.
	
	$Q_2$ ergibt sich aus der Steigung der Messkurve in \cref{fig_Kuehl}.
	Es wurde ein linearer Fit verwendet, um die Steigung nahe der Raumtemperatur zu Bestimmen.
	Die Kühlleistung $P_\text{Kühl}$ ist gegeben durch zeitliches Ableiten von \cref{eq_Wärme}.
	\begin{equation}
	\dot{Q} = P_\text{Kühl} = c \cdot m \cdot s = c \cdot \rho \cdot V \cdot s
	\label{eq_Kühlleistung}
	\end{equation}
	
	Mittels Division der Kühlleistung $P_\text{Kühl}$ durch die Frequenz des Motors erhält man die Wärme $Q_2$ pro Umlauf:
	\begin{equation}
	Q_2 = \frac{P_\text{Kühl}}{f} = c \cdot \rho \cdot V \cdot \frac{s}{f}	
	\end{equation}
	In dem Reagenzglas im Zylinderkopf befand sich $V$=\SI{1+- 0,007}{mL} destilliertes Wasser und die Steigung $s$ beträgt $\SI{0,2002+-0,0002}{\degreeCelsius/s}$. %TODO Satz
	Folglich ist $P_\text{Kühl}$ = \SI{0,838+-0,006}{W} und $Q_2$ = $\SI{0,266+-0,002}{J}$.
	
	Die Leistungszahl $\epsilon$ ist der Quotient der entnommenen Wärmemenge $Q_2$ und aufgewandter Arbeit $W$:
	\begin{equation}
		\epsilon = \frac{|Q_2|}{|W|}
		\label{eq_Leistungszahl}
	\end{equation}
	Hier beträgt $W=Q_1+W_R-Q_2=\SI{5,26+-0,33}{J}$ woraus eine Lesitungszahl von $\epsilon$ = $\SI{5,1 +- 0,3}{\%}$ folgt.
	
	\begin{figure}[H]
		\includegraphics[width=0.7\textwidth]{Kuehl}
		\centering
		\caption{Gemessene Temperatur als Funktion der Zeit beim betreiben des Striling-Motors als Kältemaschine. Die Fehler sind kleiner als die Symbole.}
		\label{fig_Kuehl}
		\centering
	\end{figure}

	Unter der Annahme, dass der Motor der Probe konstant Wärme entzieht, lässt sich die von dem Wasser beim Schmelzen abgegebene Wärme anhand von \cref{fig_Kuehl} abschätzen.
	Im Zeitraum von \SI{266+-2}{s} bis \SI{545+-5}{s} entzieht der Motor dem Wasser Wärme, die Anfangs- und Endtemperatur sind jedoch gleich. 
	Folglich entspricht die entzogene Wärme pro Masse der Schmelzwärme $S$ = \SI{233,7+-4.5}{J/g}, gemäß:
	\begin{equation}
	S = \frac{Q}{m} = c \cdot s\cdot \Delta{t}
	\end{equation}
	
	
	
	\subsubsection{Bestimmung der Heizleistung}
	Die Heizleistung lässt sich analog zur Kühlleistung in \cref{sssec_Kühlleistung} aus der Steigung $s_\text{l}$ der Messkurve bei Raumtemperatur in \cref{fig_Waerm} bestimmen.
	
	Die Steigung $s_\text{l}$ beträgt \SI{0,377+-0,001}{\degreeCelsius/s} somit folgt aus \cref{eq_Kühlleistung} eine Heizleistung $P_\text{Heiz}$ von \SI{1,58+-0,01}{W}.
	%TODO CHECK START
	Analog lässt sich auch die Leistungszahl $\epsilon$ bestimmen. 
	Die Temperaturänderung $\Delta{T}$ betrug $\SI{0,3+-0,05}{\degreeCelsius}$ woraus gemäß \cref{eq_Reibungsarbeit} ein $Q_1+W_R$ von $\SI{1,66+-0,28}{J}$ folgt. 
	$Q_2$ ergibts sich aus $P_\text{Heiz}/f$=\SI{0,501+-0,004}{J}. %TODO CREFS auf gleichungen
	Die Leistungszahl beträgt somit \SI{43,1+-10,4}{\%}.
	\begin{figure}[H]
		\includegraphics[width=0.7\textwidth]{Waerm}
		\centering
		\caption{Gemessene Temperatur als Funktion der Zeit beim betreiben des Striling-Motors als Wärmepumpe. Die Fehler sind kleiner als die Symbole.}
		\label{fig_Waerm}
		\centering
	\end{figure}
	%TODO CEHCK END
	Die spezifische Wärme von Eis lässt sich aus \cref{eq_Kühlleistung} bestimmen:
	\begin{equation}
		c_\text{Eis} = \frac{P_\text{Heiz}}{m \cdot s_\text{s}} = c_{H_2O} \frac{s_\text{l}}{s_\text{s}}
	\end{equation} 
	Die Steigung $s_\text{s}$ nimmt einen Wert von \SI{0,700+-0,001}{\degreeCelsius/s} im Bereich von \SI{-25}{\degreeCelsius} bis \SI{-5}{\degreeCelsius} an. 
	Es ergibt sich eine spezfische Wärme für Eis $c_\text{Eis}$ von $\SI{2,254+-0,007}{J/g/K}$.
	\subsection{Diskussion}
	%TODO Bezug/Nutzten oder sonst was
	%TODO auch hier die Hypothese wiederholen
	%TODO keine Messwerte hier, nach manchen Menschen, zumindest "direkt" erstellte Diagramme net hier, auch wenn Lesbarkeit-bla
	
	%TODO linear fit bei Raum temp, da sonst stochastisch schmelz?
	%TODO Schmelzwärme kleiner weil Umgebung, z.B,. Kondenzwasser
	
	
	%TODO Schmilz Wasser bla eigentliche Beobachtungen
	
	\section{Schlussfolgerung}
	%TODO Rückgriff auf Hypothese und drittes Nennen dieser
	
	%TODO Quellen zitieren, Websiten mit Zugriffsdatum
	%TODO Verweise auf das Laborbuch (sind erlaubt)
	%TODO Tabelle + Bilder mit Beschriftung
	%\printbibliography
\end{document}
