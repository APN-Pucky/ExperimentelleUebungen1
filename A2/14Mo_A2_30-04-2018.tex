% Autor: Leonhard Segger, Alexander Neuwirth
% Datum: 2017-10-30
\documentclass[
	% Papierformat
	a4paper,
	% Schriftgröße (beliebige Größen mit „fontsize=Xpt“)
	12pt,
	% Schreibt die Papiergröße korrekt ins Ausgabedokument
	pagesize,
	% Sprache für z.B. Babel
	ngerman
]{scrartcl}

% Achtung: Die Reihenfolge der Pakete kann (leider) wichtig sein!
% Insbesondere sollten (so wie hier) babel, fontenc und inputenc (in dieser
% Reihenfolge) als Erstes und hyperref und cleveref (Reihenfolge auch hier
% beachten) als Letztes geladen werden!

% Silbentrennung etc.; Sprache wird durch Option bei \documentclass festgelegt
\usepackage{babel}
% Verwendung der Zeichentabelle T1 (Sonderzeichen etc.)
\usepackage[T1]{fontenc}
% Legt die Zeichenkodierung der Eingabedatei fest, z.B. UTF-8
\usepackage[utf8]{inputenc}
% Schriftart
\usepackage{lmodern}
% Zusätzliche Sonderzeichen
\usepackage{textcomp}

% Mathepaket (intlimits: Grenzen über/unter Integralzeichen)
\usepackage[intlimits]{amsmath}
% Ermöglicht die Nutzung von \SI{Zahl}{Einheit} u.a.
\usepackage{siunitx}
% Zum flexiblen Einbinden von Grafiken (\includegraphics)
\usepackage{graphicx}
% Abbildungen im Fließtext
\usepackage{wrapfig}
% Abbildungen nebeneinander (subfigure, subtable)
\usepackage{subcaption}
% Funktionen für Anführungszeichen
\usepackage{csquotes}
% Zitieren, Bibliographie
\usepackage{biblatex}


% Zur Darstellung von Webadressen
\usepackage{url}
%chemische Formeln
\usepackage[version=4]{mhchem}
% siunitx: Deutsche Ausgabe, Messfehler getrennt mit ± ausgeben
\usepackage{floatrow}
\floatsetup[table]{capposition=top}
\usepackage{float}
% Verlinkt Textstellen im PDF-Dokument
\usepackage[unicode]{hyperref}
% "Schlaue" Referenzen (nach hyperref laden!)
\usepackage{cleveref}
\sisetup{
	locale=DE,
	separate-uncertainty
}
\bibliography{14Mo_A2_30-04-2018_References}

\begin{document}
	
	\begin{titlepage}
		\centering
		{\scshape\LARGE Versuchsbericht zu \par}
		\vspace{1cm}
		{\scshape\huge A2 - Franck-Hertz-Versuch \par} 
		\vspace{2.5cm}
		{\LARGE Gruppe 14Mo \par}
		\vspace{0.5cm}
		
		{\large Alexander Neuwirth (E-Mail: a\_neuw01@wwu.de) \par}
		{\large Leonhard Segger (E-Mail: l\_segg03@uni-muenster.de) \par}
		\vfill
		
		durchgeführt am 30.04.2018\par
		betreut von\par
		{\large Fabian Schöttke}
		
		\vfill
		
		{\large \today\par}
	\end{titlepage}
	\tableofcontents
	\newpage

	%TODO mehr TODO in Default	

	\section{Kurzfassung}
	%TODO Hypothese	und deren Ergebnis
	%TODO Ergebnisse, auch Zahlen, mindestens wenn's halbwegs Sinn ergibt
	%TODO Was wurde gemacht
	Der Franck-Hertz-Versuch erlaubt es die gequantelten Energieniveaus in Atomen nachzuweisen, da sich bei passender mittlerer freier Weglänge zwischen den Atomen eine charakteristische $I_A/U_B$-Kurve messen lässt.
	Diese ist nur erklärbar, wenn die Übergänge der Elektronen zwischen den Energieniveaus diskrete Energiedifferenzen bedeuten.
	
	Dementsprechend erwarteten wir bei einer Neonröhre und bei aufgeheizter Quecksilberdampfröhre die Franck-Hertz-Kennlinie der Röhren mit mehreren Extremwerten.
	Die mittlere freie Weglänge, die sich aus den Extrema bestimmen lässt, sollte kleiner sein als der Abstand zwischen Kathode und Anode, was sich so zeigen ließ.
	Die Wellenlängen der erzeugten Strahlung erwarteten wir gemäß den wahrscheinlichen Übergängen in den Termschemata von Quecksilber und Neon.
	Tatsächlich haben wir aber im Fall von Neon einen der Übergänge, der nicht im Bereich des sichtbaren Lichts liegt, gemessen, aber in beiden Fällen bestätigen unsere Messwerte einen der Übergänge im jeweiligen Atom.
	Die Emission von sichtbarem Licht, war jedoch bei der Neonröhre erwartungsgemäß beobachtbar.
	
	Bei kalter Quecksilberröhre wurde hingegen ein monotoner, plötzlicher Anstieg des Anodenstroms mit der Beschleunigungsspannung erwartet, da hier die mittlere freie Weglänge zu groß ist, um eine nennenswerte Anzahl von inelastischen Zusammenstößen zu ermöglichen und somit keine Minima auftreten.
	Dies konnte aufgrund eines Fehlers in der Messapparatur weder nachgewiesen noch widerlegt werden.
	
	\section{Methoden}
	Untersucht wurde eine Franck-Hertz-Röhre mit Quecksilberfüllung und eine mit Neonfüllung.
	Diese wurden, wie in  \cref{Roehren_Schaltung} dargestellt, verschaltet. %ist das schön?
	Die Quecksilberröhre befand sich in einem Ofen, der sie auf bis zu \SI{300}{\degreeCelsius} aufheizen kann.
	Der Anodenstrom ist sehr klein, weshalb er vom Betriebsgerät in eine Spannung $U_A$ umgewandelt wurde, die zum Anodenstrom proportional ist.
	
	Zunächst wurde die $I_A/U_B$-Charakteristik der Röhre mit Quecksilberfüllung bei Zimmertemperatur aufgenommen.
	Dazu wurde die Beschleunigungsspannung $U_B$ langsam erhöht und diese sowie die Spannung $U_A$ gemessen.
	
	Im Anschluss wurde der Ofen auf ca. \SI{180}{\degreeCelsius} erhitzt.
	Dann wurde das Betriebsgerät so eingestellt, dass es eine Dreieckspannung mit einer Frequenz von \SI{60}{\hertz} als Beschleunigungsspannung ausgibt.
	Der resultierende Anodenstrom wurde zunächst mit einem Oszilloskop betrachtet und Bremsspannung $U_B$ und Heizstrom $I_H$ so eingestellt, dass sich mindestens drei Minima der Franck-Hertz-Kurve ablesen ließen.
	Dann wurde mithilfe manueller Reglung der Beschleunigungsspannung die $I_A/U_B$-Charakteristik wie zuvor aufgenommen und die Temperatur im Ofen gemessen.
	
	Analog wurde die Neon-Röhre bei Raumtemperatur untersucht, wobei hier zusätzlich ein Steuergitter (mit Spannung $U_S$) verwendet wurde, um störende Einflüsse durch Abstoßung der Elektronen untereinander zu verringern. %TODO Ist das so korrekt?
	%TODO oszi
	\begin{figure}[H]
		\includegraphics[width=0.7\textwidth]{Roehren}
		\centering
		\caption{Schaltungen der Franck-Hertz-Röhren mit Quecksilber (links) und Neon (rechts).\cite{Roehren} }
		\label{Roehren_Schaltung}
		\centering
	\end{figure}

	\section{Ergebnisse und Diskussion}
	%TODO Datenanalyse -> Überschrift?
	%TODO Unsicherheiten
	

	\subsection{Beobachtung}
	%TODO Komplikationen mit MEssgerät => Trioden linie graph => ist Falsch 
	
	%TODO Mann misst auch eine Spannung selbst wenn keine Beschleunigungs spannung angelegt ist
	
	%TODO Temperatur schwankte zwischen 165 und 180
	%TODO Oszi graphen (ggf. nur der eine)
	%TODO Erhöhung von Gegenspannung => Ausgangsspannung kleiner
	%TODO T > 190°C Verlauf => MAxima "fallen runter" ((=> kein durchkommen der e- mehr))
	%TODO Beim Neon sieht man leuchtende Linien (zuerst eine dann mehrere zusätzlich).
	%TODO Bem HG ist kein Leuchten zusehen
	%TODO Sichtbare Charakteristik /= Messungscharakteristik
	
	%TODO Gegenspannung s wert erwähnen
	
	\begin{figure}[H]
		\includegraphics[width=0.7\textwidth]{Hg19}
		\centering
		\caption{Aufgenommene Quecksilber-Charakteristik bei $T$=$\SI{19,0 +- 1,5}{\degreeCelsius}$. Die Stromstärke wurde mit einem Operationsverstärker in eine messbare Spannung umgewandelt.}
		\label{Hg19}
		\centering
	\end{figure}
	
	\subsection{Datenanalyse}
	\subsubsection{Unsicherheiten}
	Die Unsicherheit des  Voltmeters beträgt $\pm (0,5\% + \SI{200}{mV})$ für die Beschleunigungsspannung und $\pm (0,5\% + \SI{20}{mV})$ für die gemessene Spannung (0,5\% vom angezeigten Wert).\cite{FH-Pforzheim} Die zusätzliche Unsicherheit des Operationsverstärkers wird als demgegenüber vernachlässigbar angenommen. 
	
	Die Unsicherheit des Thermometers vom Typ K ist \SI{1,5}{\degreeCelsius} in dem gemessenen Temperaturinterval.\cite{DIN} 
	Zusätzlich ist die Temperatur nicht überall im Heizkasten gleich und schwankte beim Aufnehmen der Quecksilber-Charakteristik von 165 bis \SI{180}{\degreeCelsius}, deshalb wählen wir für diese Messung die Unsicherheit als \SI{7}{\degreeCelsius}.
	
	Bei der Bestimmung der Beschleunigungsspannung an Extremstellen nehmen wir die Unsicherheit als aus dem Verlauf der Kurve und dem Abstand zum nächsten Messpunkt zusammengesetzt an.
	
	\subsubsection{Quecksilber-Charakteristik}
	In \cref{Hg175} ist die $I_A/U_B$-Charakteristik des Quecksilbers bei $T$=$\SI{175 +- 7}{\degreeCelsius}$ dargestellt. Daraus lassen sich folgende Abstände ablesen:
	\begin{itemize}
		\item Maxima:
		\begin{equation*}
			\Delta{U}_1 = \SI{27,1 +- 0,3}{V} -\SI{21 +- 0,1}{V} = \SI{6,1 +- 0,3}{V}
		\end{equation*}
		\item Minima
		\begin{equation*}\Delta{U}_2 = \SI{29,4 +- 0,2}{V} -\SI{24,1 +- 0,2}{V} = \SI{5,3 +- 0,3}{V}\end{equation*}
		\begin{equation*}\Delta{U}_3 = \SI{24,1 +- 0,2}{V}- \SI{18,0 +- 0,5}{V} = \SI{6,1 +- 0,5}{V}\end{equation*}
	\end{itemize}
	Im Mittel ergibt sich ein $\Delta{U_\text{Hg}}$ von $\SI{5,8 +- 0,2}{V}$.
	
	\begin{figure}[H]
		\includegraphics[width=0.7\textwidth]{Hg175}
		\centering
		\caption{Aufgenommene Quecksilber-Charakteristik bei $T$=$\SI{175 +- 7}{\degreeCelsius}$. Die Stromstärke wurde mit einem Operationsverstärker in eine messbare Spannung umgewandelt.}
		\label{Hg175}
		\centering
	\end{figure}


	\subsubsection{Neon-Charakteristik}
	In \cref{Ne19} ist die $I_A/U_B$-Charakteristik des Neons bei $T$=$\SI{19 +- 1,5}{\degreeCelsius}$ dargestellt. Daraus lassen sich folgende Abstände ablesen:
	\begin{itemize}
		\item Maxima:
		\begin{equation*}
		\Delta{U}_1 = \SI{38,8 +- 0,2}{V} -\SI{20,8 +- 0,4}{V} = \SI{18,0+- 0,4}{V}
		\end{equation*}
		\begin{equation*}
		\Delta{U}_2 = \SI{57,2 +- 0,2}{V} - \SI{38,8 +- 0,2}{V} = \SI{18,4 +- 0,3}{V}
		\end{equation*}
		\item Minima
		\begin{equation*}\Delta{U}_3 = \SI{44,9 +- 0,5}{V} -\SI{27,5 +- 0,3}{V} = \SI{17,4 +- 0,6}{V}\end{equation*}
		\begin{equation*}\Delta{U}_4 = \SI{62,9 +- 0,5}{V}- \SI{45,5 +- 0,4}{V} = \SI{18,0 +- 0,7}{V}\end{equation*}
	\end{itemize}
	Im Mittel ergibt sich ein $\Delta{U_\text{Ne}}$ von $\SI{17,9 +- 0,3}{V}$.
	\begin{figure}[H]
		\includegraphics[width=0.7\textwidth]{Ne19}
		\centering
		\caption{Aufgenommene Neon-Charakteristik bei $T$=$\SI{19,0 +- 1,5}{\degreeCelsius}$. Die Stromstärke wurde mit einem Operationsverstärker in eine messbare Spannung umgewandelt.}
		\label{Ne19}
		\centering
	\end{figure}

	\subsubsection{Bestimmen von Anregungsenergie, Wellenlänge und Frequenz der Strahlung}
	Aus den Spannungen lässt sich die kinetische Energie eines Elektrons bestimmen, die nowendig ist um den Resonanzzustand des Atoms anzuregen.
	Sie beträgt $\Delta{E}=\Delta{U}e$.
	Die Frequenz folgt aus $\nu=\Delta{E}/h$ und die Wellenlänge aus $\lambda=c/\nu$.\cite{NIST}
	Die jeweiligen Werte sind in \cref{TabelleEnergie} aufgelistet.
	\begin{table}[H]
		\centering
		\begin{tabular}{ c | c | c | c | c }
			&$\Delta{U}$ & $\Delta{E}$ &  $\nu$ & $\lambda$ \\ \hline
			Quecksilber&\SI{5,8 +- 0,2}{V} &\SI{5,8 +- 0,2}{eV} & \SI{1402 +- 48}{THz} & \SI{214,0 +- 6,3}{nm} \\
			Neon&\SI{17,9+- 0,3}{V} & \SI{17,9+- 0,3}{eV} & \SI{4328 +- 73}{THz} & \SI{69,3 +- 1,2}{nm} \\
		\end{tabular}
		\caption{Aus den Charakteristiken von Quecksilber und Neon berechnete kinetische Energie, sowie  Frequenz und Wellenlänge des emittierten Lichts.}
		\label{TabelleEnergie} 
	\end{table}
	
	\subsubsection{Berechnen der  mittleren freien Weglänge der Elektronen}
	In der Einführung wurde folgende Formel aufgeführt zum Bestimmen der freien Weglänge $\lambda$ der Elektronen:
	\begin{equation}
		\lambda = \frac{k_B T}{\sigma p} \quad \text{mit} \quad \sigma = \pi r_\text{Hg}^2
		\label{frei}
	\end{equation}
	Der Druck $p$ wird durch die Clausius-Clapeyron-Gleichung in integrierter Form bestimmt:
	\begin{equation}
		\ln\left(\frac{p_2}{p_1}\right) = \frac{\Delta{H_\text{m,v}}}{R} \left( \frac{1}{T_1} - \frac{1}{T_2} \right)
		\label{Clausius}
	\end{equation}
	Dabei beträgt die allgemeine Gaskonstante $R$ = $\SI{8,3145}{\joule \per \mol \per \kelvin}$, die Verdampfungsenthalpie von Quecksilber $\Delta{H_\text{m,v}}$ =  $\SI{59,3}{kJ/mol}$, der Radius eines Quecksilberatoms $r_\text{Hg}$ = $\SI{150}{pm}$ und der Referenzpunkt ist $T_1$ = $\SI{293,15}{K}$ mit $p_1$ = $\SI{0,242}{Pa}$.\cite{Quecksilber}\cite{Enthalpie}\cite{NIST}
	
	Durch Umformen von \cref{Clausius} ergibt sich $p_\text{kalt}$ und $p_\text{warm}$.
	Daraus widerum folgt mit \cref{frei} $\lambda_\text{kalt}$ und $\lambda_\text{warm}$.
	Die so bestimmten Werte sind in \cref{TabelleFrei} enthalten.
	
	\begin{table}[H]
		\centering
		\begin{tabular}{ c | c | c | c }
			&$T$ & $p$ &  $\lambda$ \\ \hline
			kalt&\SI{292.15 +- 1,5}{K} &\SI{0,223+-0,028}{Pa} &  \SI{0,256+- 0,032}{m} \\
			warm&\SI{453.15+- 7}{K} & \SI{1,3+-0,2}{kPa} & \SI{0,068 +- 0,010}{mm} \\
		\end{tabular}
		\caption{Mittlere frei Weglänge von Elektronen in Quecksilberdampf bei Raumtemperatur(kalt) und Heiztemperatur(warm).}
		\label{TabelleFrei} 
	\end{table}
	
	%TODO ANhang mit   Formeln vong GUM???????????????????????????????????!!!!!!!!!!!!???????????????
	
	
	\subsection{Diskussion}
	%TODO Bezug/Nutzten oder sonst was
	%TODO auch hier die Hypothese wiederholen
	Die Messung der Quecksilber-Charakteristik bei Raumtemperatur \cref{Hg19} zeigt zwar einen monotonen Anstieg des Anodenstroms mit der Beschleunigungsspannung, was zu erwarten war, da hier aufgrund der hohen mittleren Weglänge kaum inelastische Stöße zwischen Elektronen und Atomen stattfinden. %TODO ist das true mit mittl. Wegl.
	Der charakteristische steile Anstieg ab $U_B \geq U_G$ lässt sich jedoch nicht erkennen.
	Dies liegt vermutlich an einem Fehler in der Elektronik, der entstand, wenn man die Beschleunigungsspannung am dafür vorgesehenen Ausgang des Operationsverstärkers und nicht direkt zwischen Kathode und Anode misst, wie es in den anderen Messungen getan wurde.
	\newline
	Die Quecksilber-Charakteristik \cref{Hg175} zeigt die erwarteten Minima und Maxima, die durch unelastische Stöße der Elektronen entstehen, bei Elektronenenergien, die ausreichen, um die Atome anzuregen.
	
	\begin{figure}[H]
		\includegraphics[width=0.7\textwidth]{Term}
		\centering
		\caption{Vereinfachte Termschemata von Quecksilber (links) und Neon
			(rechts). Dicke Pfeile stellen Übergänge mit der größten Wahrscheinlichkeit
			dar.\cite{Roehren} }
		\label{Term}
		\centering
	\end{figure}
	

	%TODO Bei 0V Beschleunigungsspannung ggf. grundSpannugn am Operationsverstärker /= Null
	%TODO ^^^^^^ bzw bei U_b < U_g
	
	%TODO WARUM ist Ne Licht sichtbar
	
	%TODO Messung /= sichtbarem Ne, da andere Wellenlänge sichtbar vs MEssung
	
	
	%TODO WARUM verschwinden HG Charakteristiken bei T >190°C
	
	%TODO Spannung zu hoch => interne Verlusste müssen ausgeglichen werden?
	%TODO ^^^^^^^Quantisierte Energie denoch deutlich erkennbar
	
	%TODO Weglängen vergleich
	
	\section{Schlussfolgerung}
	%TODO Rückgriff auf Hypothese und drittes Nennen dieser
	
	%TODO Quellen zitieren, Websiten mit Zugriffsdatum
	%TODO Verweise auf das Laborbuch (sind erlaubt)
	%TODO Tabelle + Bilder mit Beschriftung
	\printbibliography
\end{document}
