% Autor: Leonhard Segger, Alexander Neuwirth
% Datum: 2017-10-30
\documentclass[
	% Papierformat
	a4paper,
	% Schriftgröße (beliebige Größen mit „fontsize=Xpt“)
	12pt,
	% Schreibt die Papiergröße korrekt ins Ausgabedokument
	pagesize,
	% Sprache für z.B. Babel
	ngerman
]{scrartcl}

% Achtung: Die Reihenfolge der Pakete kann (leider) wichtig sein!
% Insbesondere sollten (so wie hier) babel, fontenc und inputenc (in dieser
% Reihenfolge) als Erstes und hyperref und cleveref (Reihenfolge auch hier
% beachten) als Letztes geladen werden!

% Silbentrennung etc.; Sprache wird durch Option bei \documentclass festgelegt
\usepackage{babel}
% Verwendung der Zeichentabelle T1 (Sonderzeichen etc.)
\usepackage[T1]{fontenc}
% Legt die Zeichenkodierung der Eingabedatei fest, z.B. UTF-8
\usepackage[utf8]{inputenc}
% Schriftart
\usepackage{lmodern}
% Zusätzliche Sonderzeichen
\usepackage{textcomp}

% Mathepaket (intlimits: Grenzen über/unter Integralzeichen)
\usepackage[intlimits]{amsmath}
% Ermöglicht die Nutzung von \SI{Zahl}{Einheit} u.a.
\usepackage{siunitx}
% Zum flexiblen Einbinden von Grafiken (\includegraphics)
\usepackage{graphicx}
% Abbildungen im Fließtext
\usepackage{wrapfig}
% Abbildungen nebeneinander (subfigure, subtable)
\usepackage{subcaption}
% Funktionen für Anführungszeichen
\usepackage{csquotes}
% Zitieren, Bibliographie
\usepackage{biblatex}


% Zur Darstellung von Webadressen
\usepackage{url}
%chemische Formeln
\usepackage[version=4]{mhchem}
% siunitx: Deutsche Ausgabe, Messfehler getrennt mit ± ausgeben
\usepackage{floatrow}
\floatsetup[table]{capposition=top}
% Verlinkt Textstellen im PDF-Dokument
\usepackage[unicode]{hyperref}
% "Schlaue" Referenzen (nach hyperref laden!)
\usepackage{cleveref}
\sisetup{
	locale=DE,
	separate-uncertainty
}
%\bibliography{6Mi_S2_25-10-2017_References}

\begin{document}
	
	\begin{titlepage}
		\centering
		{\scshape\LARGE Versuchsbericht zu \par}
		\vspace{1cm}
		{\scshape\huge M3 - Elaszizität\par}
		\vspace{2.5cm}
		{\LARGE Gruppe 6Mi \par}
		\vspace{0.5cm}
		
		{\large Alexander Neuwirth (E-Mail: a\_neuw01@wwu.de) \par}
		{\large Leonhard Segger (E-Mail: l\_segg03@uni-muenster.de) \par}
		\vfill
		
		durchgeführt am 29.11.2017\par
		betreut von\par
		{\large Christian Thiede}
		
		\vfill
		
		{\large \today\par}
	\end{titlepage}
	\tableofcontents
	\newpage
	
	\section{Kurzfassung}
	***Kurzfassung\\
	Um die Elastizität verschiedener Materialien zu untersuchen, wurden zwei Experimente durchgeführt. Zunächst wurden die Auslenkungen von Stäben unter Last gemessen und daraus deren Elastizität bestimmt. Dies ließ Schlüsse auf die Art der Materialien zu. Dann wurde mithilfe verschiedener angehängter Objekte ein Torsionspendel untersucht und so der Schubmodul des Torsionsdrahtes bestimmt.  %TODO Schubmodul richtig?
	Der so ermittelte Schubmodul wurde mit dem zu erwartenden Wert für das vermutete Material des Drahtes verglichen.

	\section{Methoden}
	***Methoden \\
	*** Paralaxen frei wegen spiegel
	*** Schwing MEsspunkt bei max. Speed
	
	\subsection{Biegung Metallstäbe} %maybe besseren Namen finden
	Zunächst wurde, um den Elastizitätsmodul von verschiedenen Materialien zu bestimmen, ihre Durchbiegung in Abhängigkeit von der auf sie wirkenden Kraft gemessen. Dazu wurden vier Stäbe unterschiedlichen Materials an einem Ende waagerecht eingespannt, an ihr anderes Ende fünf verschiedene Gewichte gehängt und dann die senkrechte Auslenkung dieses Endes gemessen.
	Dabei wurde jeweils zwischen jeder Messung die Ruhelage des Stabes ohne Gewicht neu gemessen. Parallaxenfreiheit beim Ablesen der Auslenkungsskala wurde sichergestellt, indem man so über den Stab gepeilt hat, dass die Reflexion des Stabes im Spiegel hinter dem Stab verschwindet. %hoffe des is verständlich
	Dann wurden die Abmessungen der Stäbe an fünf Stellen je dreimal mit einer Mikrometerschraube  gemessen. Hierdurch wird des Fehler dieser Messung sehr gering, wenn sichergestellt ist, dass kein systematischer Fehler durch eine falsche Nullposition der Mikrometerschraube existiert. Dies wurde sichergestellt, indem die Position der Mikrometerschraube im komplett zugeschraubten Zustand überprüft wurde.
	
	\subsection{Torsionspendel}
	Der zweite Versuch bestand darin die Schwingung eines Torsionspendels zu untersuchen, um den Schubmodul des Drahtes, an dem das Pendel aufgehängt ist zu bestimmen. %TODO Schubmodul richtig?
	 Dazu wurde erst die Schwingungsdauer mit angehängter zylindrischer Scheibe gemessen und der Durchmesser des Torsionsdrahtes an fünf verschiedenen Stellen je drei mal gemessen. Dann wurde noch Höhe, Durchmesser und Masse der Scheibe bestimmt.
	Daraufhin wurde die Schwingung des Torsionspendel mit angehängter Hantel untersucht. Hierzu wurde zunächst die Schwingungsdauer der Hantel ohne aufgelgte Scheiben und dann mit zwei Scheiben, die sich in fünf verschiedenen Abständen vom Schwerpunkt der Hantelachse befanden, gemessen. Die Abmessungen und die Masse der Hantel sowie der Scheiben wurde ebenfalls festgestellt. Die Massen waren auf den betreffenden Teilen angegeben.
	Die Länge des Torsionsdrahtes wurde ebenfalls gemessen und in allen Fällen wurde eine Anfangsauslenkung von etwa 180° verwandt. 
	
	\section{Ergebnisse und Diskussion}

	\subsection{Beobachtung}
	*** linear-> Fit und Algorithmus angeben vgl. Theorie\\
	*** Biegungen in einen Graphen \\
	*** Graphen beschreiben \\
	*** Unsicherheitenrechnung \\

	\subsection*{Daten}
	\subsection*{Unsicherheiten}
	\begin{table}[H]
	\centering
	\begin{tabular}{ l | c | c | c | c |}
		& Mikroschraube  & Massband/Biegungsanzeige & Stoppuhranzeige & Reaktionszeit \\ \hline
		Unsicherheit  & $\SI{0,00577}{cm}$ &  $\SI{0,05774}{cm}$ &  $\SI{0,005774}{s}$ &  $\SI{0,11547}{s}$  \\ \hline
	\end{tabular}
	\caption{Unsicherheiten der verwendeten Messinstrumente. Die Wahrscheinlichkeitsdichtefunktionen wurden als rechteckig angenommen.}
		\label{TabelleStatischKopplungsgrad}
	\end{table}
	
	
	\subsection{Diskussion}
	*** Gewichte als exakt angenommen
	***Materialien vergleich mit Literatur
	
	\section{Schlussfolgerung}
	***Materialien vergleich mit Literatur \\
	
	
	%\printbibliography
\end{document}
