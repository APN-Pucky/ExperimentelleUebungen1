% Autor: Leonhard Segger, Alexander Neuwirth
% Datum: 2017-10-30
\documentclass[
	% Papierformat
	a4paper,
	% Schriftgröße (beliebige Größen mit „fontsize=Xpt“)
	12pt,
	% Schreibt die Papiergröße korrekt ins Ausgabedokument
	pagesize,
	% Sprache für z.B. Babel
	ngerman
]{scrartcl}

% Achtung: Die Reihenfolge der Pakete kann (leider) wichtig sein!
% Insbesondere sollten (so wie hier) babel, fontenc und inputenc (in dieser
% Reihenfolge) als Erstes und hyperref und cleveref (Reihenfolge auch hier
% beachten) als Letztes geladen werden!

% Silbentrennung etc.; Sprache wird durch Option bei \documentclass festgelegt
\usepackage{babel}
% Verwendung der Zeichentabelle T1 (Sonderzeichen etc.)
\usepackage[T1]{fontenc}
% Legt die Zeichenkodierung der Eingabedatei fest, z.B. UTF-8
\usepackage[utf8]{inputenc}
% Schriftart
\usepackage{lmodern}
% Zusätzliche Sonderzeichen
\usepackage{textcomp}

% Mathepaket (intlimits: Grenzen über/unter Integralzeichen)
\usepackage[intlimits]{amsmath}
% Ermöglicht die Nutzung von \SI{Zahl}{Einheit} u.a.
\usepackage{siunitx}
% Zum flexiblen Einbinden von Grafiken (\includegraphics)
\usepackage{graphicx}
% Abbildungen im Fließtext
\usepackage{wrapfig}
% Abbildungen nebeneinander (subfigure, subtable)
\usepackage{subcaption}
% Funktionen für Anführungszeichen
\usepackage{csquotes}
\MakeOuterQuote{"}
% Zitieren, Bibliographie
\usepackage{biblatex}


% Zur Darstellung von Webadressen
\usepackage{url}
%chemische Formeln
\usepackage[version=4]{mhchem}
% siunitx: Deutsche Ausgabe, Messfehler getrennt mit ± ausgeben
\usepackage{floatrow}
\floatsetup[table]{capposition=top}
\usepackage{float}
% Verlinkt Textstellen im PDF-Dokument
\usepackage[unicode]{hyperref}
% "Schlaue" Referenzen (nach hyperref laden!)
\usepackage{cleveref}
\sisetup{
	locale=DE,
	separate-uncertainty
}
%\bibliography{14Mo_O2_11-06-2018_References}

\begin{document}
	
	\begin{titlepage}
		\centering
		{\scshape\LARGE Versuchsbericht zu \par}
		\vspace{1cm}
		{\scshape\huge O2 - Mikrowellen \par}
		\vspace{2.5cm}
		{\LARGE Gruppe 14Mo \par}
		\vspace{0.5cm}
		
		{\large Alexander Neuwirth (E-Mail: a\_neuw01@wwu.de) \par}
		{\large Leonhard Segger (E-Mail: l\_segg03@uni-muenster.de) \par}
		\vfill
		
		durchgeführt am 11.06.2018\par
		betreut von\par
		{\large Stephan Majert}
		
		\vfill
		
		{\large \today\par}
	\end{titlepage}
	\tableofcontents
	\newpage

	%TODO mehr TODO in Default	

	\section{Kurzfassung}
	%TODO Hypothese	und deren Ergebnis, wenn Hypothese ist, dass nur Theorie erfüllt, sagen: Erwartung: Theorie aus einführung (mit reflink) erfüllt
	%TODO Ergebnisse, auch Zahlen, mindestens wenn's halbwegs Sinn ergibt
	%TODO Was wurde gemacht
	%TODO manche leute wollen Passiv oder "man", manche nicht
	
	\section{Methoden}
	%TODO Bilder von der Website klauen
	%TODO Polarisationscheck ?
	
	\section{Ergebnisse und Diskussion}
	%TODO Unsicherheiten
	

	%TODO Einflüsse von veränderten Parametern auf Messung
	\subsection{Beobachtungen und Datenanalyse}
	\subsubsection{Unsicherheiten} %TODO GGF IN DATENANYLSY
	Die Unsicherheiten wurden gemäß GUM ermittelt. 
	Außerdem wurde für Unsicherheitsrechnungen die Python Bibliothek "uncertainties" verwendet.
	\begin{description}
		\item[Messschieber:] Die Unsicherheit der Messschieber wurde auf \SI{0,2}{cm}  abgeschätzt (dreieckige WDF).
		\item[Winkelmessung:]  Die Winkel wurden analog abgelesen wobei die Unsicherheit \SI{1}{\degree} beträgt. Beim Verstellen des Winkelmessarmes hat dieser jedoch viel Spielraum gehabt in dem sich der Winkelzeiger nicht verändert hat, deshalb wurde für diese Messung eine doppplte Unsicherheit gewählt.
		\item[Multimeter:] Das Multimeter zeigte die Spannung mit 2 Nachkommastellen an, sodass die Unsicherheit \SI{0,03}{V} beträgt (rechteckige WDF).
	\end{description}

	\subsubsection{Bestimmung des Quellflecks des Senders}
	Es wurde in 4 verschiedenen Abständen zum Sender 11 orthogonal die Strahlung gemessen. Die gemessenen Strahlenprofile sind in \crefrange{fig_74cm}{fig_114cm} dargestellt. 
	\begin{figure}[H]
		\includegraphics[width=1\textwidth]{fig_74cm}
		\centering
		\caption{Strahlenprofil im Abstand von \SI{74}{cm}. Die Unsicherheiten sind kleiner als die Symbole.}
		\label{fig_74cm}
		\centering
	\end{figure}

	\begin{figure}[H]
		\includegraphics[width=1\textwidth]{fig_94cm}
		\centering
		\caption{Strahlenprofil im Abstand von \SI{94}{cm}. Die Unsicherheiten sind kleiner als die Symbole.}
		\label{fig_94cm}
		\centering
	\end{figure}
	\begin{figure}[H]
		\includegraphics[width=1\textwidth]{fig_104cm}
		\centering
		\caption{Strahlenprofil im Abstand von \SI{104}{cm}. Die Unsicherheiten sind kleiner als die Symbole.}
		\label{fig_104cm}
		\centering
	\end{figure}
	\begin{figure}[H]
		\includegraphics[width=1\textwidth]{fig_114cm}
		\centering
		\caption{Strahlenprofil im Abstand von \SI{114}{cm}. Die Unsicherheiten sind kleiner als die Symbole.}
		\label{fig_114cm}
		\centering
	\end{figure}
	Um die Abweichung des Quellflecks des Senders zu seiner Position zu bestimmen wurde ein Lorentzkurven-Fit durchgeführt. Die resultierenden $xc$ in den Graphen sind in \cref{tab_xc} aufgeführt:
	\begin{table}[H]
		\centering
		\begin{tabular}{ c | c }
			Abstand Sender/Empfänger in \SI{}{cm} & $xc$ in \SI{}{cm}  \\ \hline
			\SI{74}{}&\SI{0,00+-0,17}{}\\
			\SI{94}{}&\SI{0,02+-0,11}{}\\
			\SI{104}{}&\SI{0,07+-0,13}{}\\
			\SI{114}{}&\SI{0,29+-0,13}{}\\
		\end{tabular}
		\caption{Zur Strahlungsrichtung orthogonale Abweichung $xc$ des Quellpunkts zum Sender.}
		\label{tab_xc} 
	\end{table}
	Unter der Annahme, dass die Strahlung genähert in der Tischebene eine Kugelwelle ist, nimmt die Intensität mit $1/r^2$ ab. 
	Deshalb wurde in \cref{fig_y0cm} die Fitfunktion $f(x)=a*(1/(x+b)^2)$ berechnet. $b=\SI{-20,55}{1,95}$ ist die Verschiebung des Quellflecks zur Position des Senders, d.h. der Quellfleck befindet sich vor dem Sender.
	\begin{figure}[H]
		\includegraphics[width=1\textwidth]{fig_y0cm}
		\centering
		\caption{Die Maxima der Strahlenprofile sind gegen den X-Achsen Abstand des Sensors aufgetragen. Die Unsicherheiten sind kleiner als die Symbole.}
		\label{fig_y0cm}
		\centering
	\end{figure}

	
	\subsubsection{Bestimmung der Wellenlänge}
	Bei der bestmöglichen Auflösung der Knoten der stehenden Welle betrug die Spannung noch \SI{0,03}{V}. In \cref{fig_steh_welle} sind die Positionen der Minima aufgelistet.
	Die gemessenen Positionen der jeweiligen Strahlungsminima wurden linear angeordnet. 
	Die gemittelte halbe Wellenlänge ergibt sich aus de Abstand des ersten zum zweiten Minimum geteilt durch die Anzahl der Minima. 
	So folgt: 
	\begin{equation}
		\lambda = 2 \cdot \frac{\SI{28,1 +- 0,2}{cm} + \SI{12,4+-0,2}{cm}}{10} = \SI{3,14 +- 0,05}{cm}.
	\end{equation}
	\begin{figure}[H]
		\includegraphics[width=1\textwidth]{fig_steh_welle}
		\centering
		\caption{Die gemessenen Positionen der Knoten der stehenden Welle sind dargestellt. Die Unsicherheiten sind kleiner als die Symbole.}
		\label{fig_steh_welle}
		\centering
	\end{figure}
	\subsection{Diskussion}
	%TODO Bezug/Nutzen oder sonst was
	%TODO auch hier die Hypothese wiederholen
	%TODO keine Messwerte hier, nach manchen Menschen, zumindest "direkt" erstellte Diagramme net hier, auch wenn Lesbarkeit-bla

	%TODO fraglich ob 1/r^2 von Maximums den Quellfleck genau bestimmt, da das mehr fürs Nahfelddd gilt und der Sender nicht bei nem Abstand von 0, Unendlich WElleintensität pumpen kann.
	
	\section{Schlussfolgerung}
	%TODO Rückgriff auf Hypothese und drittes Nennen dieser
	
	%TODO Quellen zitieren, Websiten mit Zugriffsdatum
	%TODO Verweise auf das Laborbuch (sind erlaubt)
	%TODO Tabelle + Bilder mit Beschriftung
	%\printbibliography
\end{document}
