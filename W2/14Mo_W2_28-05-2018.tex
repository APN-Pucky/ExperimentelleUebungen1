% Autor: Leonhard Segger, Alexander Neuwirth
% Datum: 2017-10-30
\documentclass[
	% Papierformat
	a4paper,
	% Schriftgröße (beliebige Größen mit „fontsize=Xpt“)
	12pt,
	% Schreibt die Papiergröße korrekt ins Ausgabedokument
	pagesize,
	% Sprache für z.B. Babel
	ngerman
]{scrartcl}

% Achtung: Die Reihenfolge der Pakete kann (leider) wichtig sein!
% Insbesondere sollten (so wie hier) babel, fontenc und inputenc (in dieser
% Reihenfolge) als Erstes und hyperref und cleveref (Reihenfolge auch hier
% beachten) als Letztes geladen werden!

% Silbentrennung etc.; Sprache wird durch Option bei \documentclass festgelegt
\usepackage{babel}
% Verwendung der Zeichentabelle T1 (Sonderzeichen etc.)
\usepackage[T1]{fontenc}
% Legt die Zeichenkodierung der Eingabedatei fest, z.B. UTF-8
\usepackage[utf8]{inputenc}
% Schriftart
\usepackage{lmodern}
% Zusätzliche Sonderzeichen
\usepackage{textcomp}

% Mathepaket (intlimits: Grenzen über/unter Integralzeichen)
\usepackage[intlimits]{amsmath}
% Ermöglicht die Nutzung von \SI{Zahl}{Einheit} u.a.
\usepackage{siunitx}
% Zum flexiblen Einbinden von Grafiken (\includegraphics)
\usepackage{graphicx}
% Abbildungen im Fließtext
\usepackage{wrapfig}
% Abbildungen nebeneinander (subfigure, subtable)
\usepackage{subcaption}
% Funktionen für Anführungszeichen
\usepackage{csquotes}
\MakeOuterQuote{"}
% Zitieren, Bibliographie
\usepackage{biblatex}


% Zur Darstellung von Webadressen
\usepackage{url}
%chemische Formeln
\usepackage[version=4]{mhchem}
% siunitx: Deutsche Ausgabe, Messfehler getrennt mit ± ausgeben
\usepackage{floatrow}
\floatsetup[table]{capposition=top}
\usepackage{float}
% Verlinkt Textstellen im PDF-Dokument
\usepackage[unicode]{hyperref}
% "Schlaue" Referenzen (nach hyperref laden!)
\usepackage{cleveref}
\sisetup{
	locale=DE,
	separate-uncertainty
}
%\bibliography{6Mi_M3_29-11-2017_References}
%TODO anpassen

\begin{document}
	
	\begin{titlepage}
		\centering
		{\scshape\LARGE Versuchsbericht zu \par}
		\vspace{1cm}
		{\scshape\huge W2 - Adiabatenexponent $c_p/c_v$ von Gasen \par} %mit c_p/c_v oder ohne?
		\vspace{2.5cm}
		{\LARGE Gruppe 14Mo \par}
		\vspace{0.5cm}
		
		{\large Alexander Neuwirth (E-Mail: a\_neuw01@wwu.de) \par}
		{\large Leonhard Segger (E-Mail: l\_segg03@uni-muenster.de) \par}
		\vfill
		
		durchgeführt am 28.05.2018\par
		betreut von\par
		{\large Pascal Grenz}
		
		\vfill
		
		{\large \today\par}
	\end{titlepage}
	\tableofcontents
	\newpage

	%TODO mehr TODO in Default	

	\section{Kurzfassung}
	%TODO Hypothese	und deren Ergebnis, wenn Hypothese ist, dass nur Theorie erfüllt, sagen: Erwartung: Theorie aus einführung (mit reflink) erfüllt
	%TODO Ergebnisse, auch Zahlen, mindestens wenn's halbwegs Sinn ergibt
	%TODO Was wurde gemacht
	%TODO manche leute wollen Passiv oder "man", manche nicht
	
	\section{Methoden} \label{sec_Methoden}
	%TODO Bilder von der Website klauen
	
	\section{Ergebnisse und Diskussion}
	%TODO Unsicherheiten
	

	% \subsection{Beobachtung}
	%TODO Einflüsse von veränderten Parametern auf Messung
	\subsection{Beobachtungen und Datenanalyse}
	\subsubsection{Unsicherheiten} %TODO GGF IN DATENANYLSY
	Die Unsicherheiten wurden gemäß GUM ermittelt. 
	Außerdem wurde für Unsicherheitsrechnungen die Python Bibliothek "uncertainties" verwendet.
	\begin{description}
		\item[Waage:] Die Waage zeigt das Gewicht mit einer Nachkommaselle an, woraus eine Unsicherheit von \SI{0,03}{g} folgt (rechteckige WDF).
		\item[Stoppuhr:] Die Zeit wurde in Sekunden mit zwei Nachkommastellen gemessen. Folglich ist die Unsicherheit \SI{0,003}{s} (rechteckige WDF), jedochat die Reaktionszeit einen größeren Einfluss, wesshalb eine Unsicherheit von \SI{0,1}{s} angenommen wird.
		\item[Messschieber:] Die Unsicherheit des Messschiebers wurde auf \SI{0,06}{mm}  abgeschätzt (dreieckige WDF).
		\item[Maßstäbe:]  Ebenfalls eine analoge Messung, wobei die Unsicherheit \SI{0,04}{cm} beträgt.
		\item[Schwingungszählung:] Beim Zählen der 100 Schwingungen wird von maximal einer Schwingung zu viel bzw. zu wenig ausgegangen, sodass die Unsicherheit \SI{0,6}{} beträgt (rechteckige WDF).
		\item[Luftdruck:] Der Umgebungsdruck wurde mit einer Unsicherheit von \SI{0,4}{kPa} ermittelt.
		\item[Glasflasche:] Auf der Glasflasche war keine Unsicherheit angegeben. Außerdem war unklar, ob das Volumen des Stöpfels mit in die Angabe von $\SI{5450}{cm^3}$ eingegeht oder nicht. Desshalb wurde die Unsicherheit des Volumens mit $\SI{30}{cm^3}$ abgeschätzt.
	\end{description}
	
	\subsubsection{Bestimmung von $\kappa$ nach Rüchardt-Flammersfeld}
	Es wurden wie in \cref{sec_Methoden} beschrieben die Zeit für 100 Schwingungen bei unterschielichen Abständen der Schellen gemessen.
	In \cref{fig_Rüc_Fla} sind die Schwingdauern von Luft, Argon und Kohlenstoffdioxid gegen den Abstand der Schellen gemessen. 
	Es wurde ein linearer York-Fit verwendet, da dieser auch die X-Fehler berücksichtigt.
	Aus den Y-Achsenabschnitten der Fit-Funktionen lassen sich die Schwingdauern für einen auf Null extrapolierten Wert des Schellenabstands bestimmen.
	Diese sind in \cref{tab_Rüc_Fla} aufeführt.
	
	\begin{figure}[H]
		\includegraphics[width=1\textwidth]{Rüc_Fla}
		\centering
		\caption{Gemessene Schwingdauern in Abhängigkeit von dem Abstand der Schellen.}
		\label{fig_Rüc_Fla}
		\centering
	\end{figure} 

	In der Einführung wurde folgende Formel zur Bestimmung es Adiabatenexponenten hergeleitet:
	\begin{equation}
	\kappa = \frac{4\pi^2mV_0}{p_0 A^2 T^2}
	\label{eq_Rüc_Fla_Kappa}
	\end{equation}
	\begin{equation}
	u(\kappa) = \kappa \sqrt{\left(\frac{u(m)}{m}\right)^2+\left(\frac{u(V_0)}{V_0}\right)^2+\left(\frac{u(p_0)}{p_0}\right)^2+\left(\frac{2u(T)}{T}\right)^2+\left(\frac{2u(A)}{A}\right)^2}
	\end{equation}
	
	Das Volumen $V_0$ setzt sich zusammen aus dem der Glasflasche $V_F$ = $\SI{5450+-30}{cm^3}$ und dem Glasrohr mit einem Radius $r$ = $\SI{0,798+-0,003}{cm}$ und einer Höhe zum Spalt $h$ = $\SI{10,05+-0.06}{cm}$.
	\begin{equation}
		V_0 = V_F + r^2 \pi h
	\end{equation}
	Somit betragen:
	\begin{itemize}
		\item Voumen $V_0 = \SI{5470+-30}{cm^3}$.
		\item Fläche $A = r^2\pi = \SI{1,998+-0,015}{cm^2}$
		\item Masse $m = \SI{7,2+-0,03}{g}$ (Messung)
		\item Umgebungsdruck $p_\text{L}=\SI{101,2+-0,4}{kPa}$ (Messung)
		\item Innendruck $p_0 = p_\text{L} + \frac{m\cdot g}{A} = \SI{101,5+-0,4}{kPa}$ 
	\end{itemize}
	In \cref{tab_Rüc_Fla} sind die berechneten Adiabatenkoeffizienten zu den jeweiligen Schwingdauern aufgelistet.
	
	\begin{table}[H]
		\centering
		\begin{tabular}{ c | c | c | c}
			&Luft & Argon  & Kohlenstoffdioxid\\ \hline
			Schwingungsdauer $T$ in s&\SI{0,533+-0,003}{}&\SI{0,506+-0,003}{} & \SI{0,557+-0,003}{}\\
			Adiabatenkoeffizient $\kappa$ &\SI{1,351+-0,028}{}&\SI{1,499+-0,031}{}&\SI{1,237+-0,025}{}\\
		\end{tabular}
		\caption{Extrapolierte Schwingdauern sowie resultierende Adiabatenkoeffizienten.}
		\label{tab_Rüc_Fla} 
	\end{table}
	
	
	
	\subsubsection{Bestimmung von $\kappa$ nach Clément-Desormes}
	In der Einführung wurde folgende Formel zur Bestimmung es Adiabatenexponenten hergeleitet:
	\begin{equation}
		\kappa = \frac{h_1}{h_1-h_3}
		\label{eq_Cle_Des_Kappa}
	\end{equation}
	\begin{equation}	
		u(\kappa) = \kappa^2\cdot \sqrt{\left(\frac{h_3}{h_1}\right)^2+1} \cdot \frac{u(h)}{h_1}
	\end{equation}
	Dabei ist $h_1$ die Höhe der Flüssigkeitssäule im Manometer nach der Erhöhung des Drucks im Gefäß und dessen folgender Temperaturausgleich mit der Umgebung. 
	$h_3$ ist die Höhe, die sich ergibt, wenn man den Druck im Gefäß an den der Umgebung anpasst und sich, unter Druckänderung, ein (adiabatischer) Temperaturgleichgewicht einstellt.
	
	In \cref{tab_Manometer} sind die Messwerte sowie folgende Adiabatenkoeffizienten aufgeführt. Es folgt ein Mittelwert für $\kappa_\text{Luft}$ von \SI{1,355+-0,004}{}.
	\begin{table}[H]
		\centering
		\begin{tabular}{ c | c | c }
			$h_1$ in \SI{}{cm} & $h_3$ in \SI{}{cm}  & $\kappa_\text{Luft}$\\ \hline
			\SI{16,64+-0,06}{}&\SI{4,35+-0,06}{} & \SI{1,354+-0,007}{}\\
			\SI{20,63+-0,06}{}&\SI{5,52+-0,06}{}& \SI{1,365+-0,006}{}\\
			\SI{25,34+-0,06}{}&\SI{6,72+-0,06}{}& \SI{1,361+-0,005}{}\\
			\SI{36,7+-0,06}{}&\SI{9,41+-0,06}{}& \SI{1,345+-0,003}{}\\
			\SI{10,98+-0,06}{}&\SI{2,84+-0,06}{}& \SI{1,349+-0,01}{}\\
		\end{tabular}
		\caption{Gemessene Höhe der Flüssigkeitssäule im Manometer und nach \cref{eq_Cle_Des_Kappa} berechnete Adiabtenexponenten $\kappa_\text{Luft}$ von Luft.}
		\label{tab_Manometer} 
	\end{table}
	\subsection{Diskussion}
	%TODO Bezug/Nutzten oder sonst was
	%TODO auch hier die Hypothese wiederholen
	%TODO keine Messwerte hier, nach manchen Menschen, zumindest "direkt" erstellte Diagramme net hier, auch wenn Lesbarkeit-bla
	
	
	%TODO Fit für Null extrapolier for Argon gf. fraglich
	%TODO Luft net optimal
	%TODO Nix Paralexen frei...
	%TODO Plastictopf ist (bissle) dehnbar /= adiabatisch
	%TODO Prüfen des Volumens Glasflasche
	%TODO check Flasche nur ein Gas
	
	\section{Schlussfolgerung}
	%TODO Rückgriff auf Hypothese und drittes Nennen dieser
	
	%TODO Quellen zitieren, Websiten mit Zugriffsdatum
	%TODO Verweise auf das Laborbuch (sind erlaubt)
	%TODO Tabelle + Bilder mit Beschriftung
	%\printbibliography
\end{document}
