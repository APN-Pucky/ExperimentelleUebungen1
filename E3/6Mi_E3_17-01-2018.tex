% Autor: Leonhard Segger, Alexander Neuwirth
% Datum: 2017-10-30
\documentclass[
	% Papierformat
	a4paper,
	% Schriftgröße (beliebige Größen mit „fontsize=Xpt“)
	12pt,
	% Schreibt die Papiergröße korrekt ins Ausgabedokument
	pagesize,
	% Sprache für z.B. Babel
	ngerman
]{scrartcl}

% Achtung: Die Reihenfolge der Pakete kann (leider) wichtig sein!
% Insbesondere sollten (so wie hier) babel, fontenc und inputenc (in dieser
% Reihenfolge) als Erstes und hyperref und cleveref (Reihenfolge auch hier
% beachten) als Letztes geladen werden!

% Silbentrennung etc.; Sprache wird durch Option bei \documentclass festgelegt
\usepackage{babel}
% Verwendung der Zeichentabelle T1 (Sonderzeichen etc.)
\usepackage[T1]{fontenc}
% Legt die Zeichenkodierung der Eingabedatei fest, z.B. UTF-8
\usepackage[utf8]{inputenc}
% Schriftart
\usepackage{lmodern}
% Zusätzliche Sonderzeichen
\usepackage{textcomp}

% Mathepaket (intlimits: Grenzen über/unter Integralzeichen)
\usepackage[intlimits]{amsmath}
% Ermöglicht die Nutzung von \SI{Zahl}{Einheit} u.a.
\usepackage{siunitx}
% Zum flexiblen Einbinden von Grafiken (\includegraphics)
\usepackage{graphicx}
% Abbildungen im Fließtext
\usepackage{wrapfig}
% Abbildungen nebeneinander (subfigure, subtable)
\usepackage{subcaption}
% Funktionen für Anführungszeichen
\usepackage{csquotes}
% Zitieren, Bibliographie
\usepackage{biblatex}


% Zur Darstellung von Webadressen
\usepackage{url}
%chemische Formeln
\usepackage[version=4]{mhchem}
% siunitx: Deutsche Ausgabe, Messfehler getrennt mit ± ausgeben
\usepackage{floatrow}
\floatsetup[table]{capposition=top}
% Verlinkt Textstellen im PDF-Dokument
\usepackage[unicode]{hyperref}
% "Schlaue" Referenzen (nach hyperref laden!)
\usepackage{cleveref}
\sisetup{
	locale=DE,
	separate-uncertainty
}
%\bibliography{6Mi_M3_29-11-2017_References}

\begin{document}
	
	\begin{titlepage}
		\centering
		{\scshape\LARGE Versuchsbericht zu \par}
		\vspace{1cm}
		{\scshape\huge E3 - Elektrische Resonanz \par} 
		\vspace{2.5cm}
		{\LARGE Gruppe 6Mi \par}
		\vspace{0.5cm}
		
		{\large Alexander Neuwirth (E-Mail: a\_neuw01@wwu.de) \par}
		{\large Leonhard Segger (E-Mail: l\_segg03@uni-muenster.de) \par}
		\vfill
		
		durchgeführt am 17.01.2018\par
		betreut von\par
		{\large Wladislaw Hartmann} %TODO Ich hoffe, das ist der richtige
		
		\vfill
		
		{\large \today\par}
	\end{titlepage}
	\tableofcontents
	\newpage

	%TODO mehr TODO in Default	

	\section{Kurzfassung}
	%TODO Hypothese	und deren Ergebnis
	%TODO Ergebnisse, auch Zahlen, mindestens wenn's halbwegs Sinn ergibt
	%TODO Was wurde gemacht
	Dinge schwingen.
	\section{Methoden}
	%TODO Bilder von der Website klauen
	
	\section{Ergebnisse und Diskussion}
	%TODO Datenanalyse -> Überschrift?
	%TODO Unsicherheiten
	\subsection{Beobachtung}
	Für die Unsicherheit des Multimeters und des Kondensators war kein Fehler angegeben, im Folgenden wird ein Fehler von 1\% des Messwerts (mit rechteckiger WDF) angenommen. %Irgendwie sieht das %-Zeichen merkwürdig aus.
	Hierbei schwankten die Messwerte kaum.
	Zusätzlich besteht eine Unsicherheit von \SI{0,02}{mV} aufgrund der Digitalanzeige.
	Diese wurde im Fall der Widerstandsmessung vernachlässigt, da sie im Vergleich zur Unsicherheit des Mulitimeters selbst verschwindet.
	In \cref{Grundwerte} sind die Messwerte des Widerstands $R_v$ bzw. $R_p$ sowie die vom Oszilloskop angezeigte Spitze-Spitze-Spannung weit entfernt und am Resonanzfall angegeben. %Spitze-Spitze-Spannung klingt scheiße, aber w/e
	Für die Unsicherheit der Spannungsangabe des Oszilloskop wurde die Schwankung des Wertes (\SI{0,04}{\volt} mit rechteckiger WDF) zugrunde gelegt und vorausgesetzt, dass diese (weil sie relativ groß war) die Unsicherheit der Anzeige und des Messgerätes ihr gegenüber verschwinden lässt.
	\begin{table}[tb]
		\centering
		\begin{tabular}{ r | c | c | c }
			& Spannung P-P fern (\si{\volt}) & Spannung P-P nah (\si{\volt}) & Widerstand (\si{\ohm})\\ \hline 
			S1& $4 \pm 0,012 $ & $3,6 \pm 0,012 $& $200 \pm 0,58$ \\ 
			S2& $4 \pm 0,012 $ & $3,72 \pm 0,012 $ & $497,4 \pm 1,44$ \\
			S3& $4 \pm 0,012 $ & $3,36 \pm 0,012 $ & $ 0 \pm 0,02 $\\
			P1& $ 10 \pm 0,012 $ & $10,04 \pm 0,012 $ & inf\\
			P2& $ 10 \pm 0,012 $ & $10,12 \pm 0,012 $ & $ \SI{9,93 \pm 0,029 e3}{} $\\
			P3& $ 10 \pm 0,012 $ & $10,04 \pm 0,012 $ & $ \SI{2,001 \pm 0,0058 e3}{} $\\
		\end{tabular}
		\caption{Einstellungen der Resonanzkreise. Dabei entspricht S1 bis S3 den Messreihen im Serienresonanzkreis und P1 bis P3 den Messungen im Parallelresonanzkreis. \enquote{Spannung P-P} bezieht sich auf die Spannung zwischen den Spitzen, die das Oszilloskop angab. \enquote{inf} meint den Fall, bei dem kein Widerstand eingesetzt wurde und keine Verbindung zwischen den Anschlüssen, die ansonsten an den Widerstand angeschlossen waren, bestand.}
		\label{Grundwerte} 
	\end{table}
	Es wurde die Frequenz mithilfe des Oszilloskops auf \SI{1000}{\hertz} eingestellt.
	Hieraus ergibt sich mit rechteckiger WDF mit einer Breite von \SI{1}{\hertz} ein Wert von \SI{1000 \pm 0,3}{\hertz}.
	Die Messung des Innenwiderstands der Spule ergab im Fall des  Serienresonanzkreises einen Wert von  \SI{52,3 \pm 0,15}{\ohm} und beim Parallelresonanzkreis \SI{19,1 \pm 0,06}{\ohm}.
	Die Unsicherheit der Spannungsmessung über dem Widerstand von \SI{10}{\ohm} beträgt die kombinierte Unsicherheit von Multimeter und Display des Multimeters.
	\par
	Die Stromstärke wurde gemäß
	\begin{equation*}
		I=\frac{U}{R}
	\end{equation*}
	mit $ R=\SI{10 \pm 0,03}{\ohm} $ (Angabe auf dem Widerstand) berechnet.
	Die Unsicherheit der Stromstärke ergab sich durch \cref{Partielle_Unsicherheiten}.
	
	
	\begin{equation}
	u(y) = \sqrt{  \sum_{i=0}^{N} \left( \frac{\partial y}{\partial x_i}u(x_i)\right)^2  }
	\label{Partielle_Unsicherheiten}
	\end{equation}
	
	\subsubsection{Serienresonanzkreis}
	In \crefrange{Serie_0}{Serie_500} ist die Stromstärke im Serienresonanzkreis gegen den Kehrwert der Kapazität bei unterschiedlichen Widerständen $ R_v $ aufgetragen.
	
	\begin{figure}[H]
		\includegraphics[width=1\textwidth]{Serie_0}
		\centering
		\caption{Hier ist die Stromstärke $ \left| I \right| $ gegen den Kehrwert der Kapazität bei $ R_v = \SI{0}{\ohm} $ aufgetragen. Die Fehlerbalken sind kleiner als die Symbolgröße.}
		\label{Serie_0}
		\centering
	\end{figure} 
	\begin{figure}[H]
		\includegraphics[width=1\textwidth]{Serie_200}
		\centering
		\caption{Hier ist die Stromstärke $ \left| I \right| $ gegen den Kehrwert der Kapazität bei $ R_v = \SI{200}{\ohm} $ aufgetragen. Die Fehlerbalken sind kleiner als die Symbolgröße.}
		\label{Serie_200}
		\centering
	\end{figure} 
	\begin{figure}[H]
		\includegraphics[width=1\textwidth]{Serie_500}
		\centering
		\caption{Hier ist die Stromstärke $ \left| I \right| $ gegen den Kehrwert der Kapazität bei $ R_v = \SI{500}{\ohm} $ aufgetragen. Die Fehlerbalken sind kleiner als die Symbolgröße.}
		\label{Serie_500}
		\centering
	\end{figure} 
	Es gilt nach der Thomsonschen Schwingungsformel:
	\begin{equation}
		\omega_0^2=\frac{1}{LC}
	\end{equation}
	Daraus ergibt sich im Fall der Variation von $C$ am Maximum $ C_\text{max} $:
	\begin{equation}
		L=\frac{1}{\omega_0^2 C_0}
		\label{Thomson}
	\end{equation}
	
	Für den Verlustwiderstand $R$ gilt für die Orte $C_i$ an denen der Strom das $ 1/\sqrt{2} $-fache des Maximums annimmt:
	\begin{equation}
		R= \frac{2}{2\omega_0 \left( \frac{1}{C_2} - \frac{1}{C_1} \right)}
	\end{equation}
	In \cref{Serie_Erg} sind die Ergebnisse aus Ablesen der Maxima und die daraus folgenden Induktivitäten der Spule sowie die Verlustwiderstände für die drei Messungen dargestellt.
	
	\begin{table}[tb]
		\centering
		\begin{tabular}{ r | c | c | c }
			&S1 & S2 & S3 \\ \hline
			$\left| I \right|_\text{max}$ & & & \\ %TODO gucken ob das so aussieht, wie es soll
			$1/C_\text{max}$ & & & \\
			$L$ & & & \\
			$R$ & & & \\
		\end{tabular}
		\caption{Abgelesene Maxima der Resonanzkurven und daraus berechnete Induktivitäten $L$ der Spule sowie Verlustwiderstände $R$ des Kreises bei den drei Messungen.}
		\label{Serie_Erg} 
	\end{table}
	
	\subsubsection{Parallelresonanzkreis}
	
	In \crefrange{Para_2k}{Para_inf} ist die Stromstärke im Serienresonanzkreis gegen die Kapazität bei unterschiedlichen Widerständen $ R_v $ aufgetragen.
	\begin{figure}[H]
		\includegraphics[width=1\textwidth]{Parallelstromkreis_inf}
		\centering
		\caption{Hier ist die Stromstärke $ \left| I \right| $ gegen die Kapazität bei $ R_v = \SI{0}{\ohm} $ aufgetragen. Die Fehlerbalken sind kleiner als die Symbolgröße.}
		\label{Para_2k}
		\centering
	\end{figure} 
	\begin{figure}[H]
		\includegraphics[width=1\textwidth]{Parallelstromkreis_10k}
		\centering
		\caption{Hier ist die Stromstärke $ \left| I \right| $ gegen die Kapazität bei $ R_v = \SI{0}{\ohm} $ aufgetragen. Die Fehlerbalken sind kleiner als die Symbolgröße.}
		\label{Para_10k}
		\centering
	\end{figure} 
	\begin{figure}[H]
		\includegraphics[width=1\textwidth]{Parallelstromkreis_inf}
		\centering
		\caption{Hier ist die Stromstärke $ \left| I \right| $ gegen die Kapazität bei $ R_v = \SI{0}{\ohm} $ aufgetragen. Die Fehlerbalken sind kleiner als die Symbolgröße.}
		\label{Para_inf}
		\centering
	\end{figure}
	
	In \cref{Para_Erg} sind die Ergebnisse aus Ablesen der Maxima und die daraus folgenden Induktivitäten der Spule sowie die Verlustwiderstände für die drei Messungen dargestellt.
	Auch hier gilt \cref{Thomson}.
	Für den Verlustwiderstand $R$ gilt für die Orte $C_i$ an denen der Strom das $ \sqrt{2} $-fache des Minimums annimmt:
	\begin{equation}
	R= \frac{2}{2\omega_0 \left( C_2 - C_1 \right)}
	\end{equation}
	
	\begin{table}[tb]
		\centering
		\begin{tabular}{ r | c | c | c }
			&S1 & S2 & S3 \\ \hline
			$  \left| I \right|_\text{min}$ & & & \\ %TODO gucken ob das so aussieht, wie es soll
			$C_\text{min}$ & & & \\
			$L$ & & & \\
			$R$ & & & \\
		\end{tabular}
		\caption{Abgelesene Minima der Resonanzkurven und daraus berechnete Induktivitäten $L$ der Spule sowie Verlustwiderstände $R$ des Kreises bei den drei Messungen.}
		\label{Para_Erg} 
	\end{table}
	\subsection{Diskussion}
	%TODO Bezug/Nutzten oder sonst was
	%TODO auch hier die Hypothese wiederholen
	
	\section{Schlussfolgerung}
	%TODO Rückgriff auf Hypothese und drittes Nennen dieser
	
	%TODO Quellen zitieren, Websiten mit Zugriffsdatum
	%TODO Verweise auf das Laborbuch (sind erlaubt)
	%TODO Tabelle + Bilder mit Beschriftung
	%\printbibliography
\end{document}
